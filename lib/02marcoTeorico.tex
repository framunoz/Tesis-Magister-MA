\chapter{Transporte Óptimo de Masas}
En este capítulo se abordará el problema de transporte óptimo, la distancia de Wasserstein, y el problema de los baricentros de Wasserstein. Además, se presentarán algunas propiedades de la distancia de Wasserstein, las cuales serán de utilidad en el desarrollo de este trabajo. La notación y definiciones utilizadas en este capítulo se encuentran basadas en \cite{villani2009optimal} y \cite{peyre2019computational}. Sin embargo, antes de empezar a enunciar definiciones y propiedades, se sentarán la notación y definiciones básicas que se utilizarán a lo largo de este trabajo.

\section{Notación}
 {
  \FM[inline]{Agregar alguna intro para lo que es una medida de probabilidad? para aquellas personas que vienen de otras áreas?}
  \begin{definition}
	  \FM[inline]{Se podría dejar notación probabilística?}
	  Se definen los siguientes espacios:
	  \begin{itemize}
		  \item $(\cX, d)$ es un espacio Polaco, si $\cX$ es un espacio métrico, completo y separable.
		  \item $\ProbSpace[\cX]$ denotará al conjunto de medidas de probabilidad en $\cX$, utilizando la $\sigma$-álgebra de Borel.
		  \item $\ProbSpaceAC[\cX]$ denotará al conjunto de medidas de probabilidad absolutamente continuas con respecto a una medida de referencia $\lambda$ (como por ejemplo, la de Lebesgue o la cuenta puntos), utilizando la $\sigma$-álgebra de Borel.
		  \item $\mathcal{C}(\cX)$ denotará al conjunto de funciones continuas en $\cX$.
		  \item ${\Lip}_{k}(\cX)$ denotará al conjunto de funciones $k$-Lipschitz en $\cX$. Mientras que se asumirá que $\Lip[\cX]$ denotará al conjunto de funciones $1$-Lipschitz en $\cX$. \FM[inline]{Se podría agregar una definición de función Lipschitz?}
	  \end{itemize}
  \end{definition}

  \begin{definition}
	  Se definirá el \emph{simplex} de dimension $n$ como el conjunto de vectores de $\R^{n}$ cuyas componentes suman 1, es decir,
	  \begin{equation}
		  \Simplex \eqdef \left\{
		  x \in [0, 1]^{n} \colon \sum_{i=1}^{n} x_{i} = 1
		  \right\},
	  \end{equation}
	  y a los elementos pertenecientes al simplex se les llamará \emph{vectores de probabilidad}.
  \end{definition}

  \begin{definition}
	  Dados $\mu \in \ProbSpace[\cX]$ y $\nu \in \ProbSpace[\cY]$, se denotará por $\Cpl(\mu, \nu)$ al conjunto de medidas de probabilidad en $\cX \times \cY$ cuyas proyecciones marginales sean $\mu$ y $\nu$, es decir,
	  \begin{equation}
		  \Cpl(\mu, \nu) \eqdef \left\{
		  \gamma \in \ProbSpace[\cX \times \cY] \colon \gamma(A \times \cY) = \mu(A), \gamma(\cX \times B) = \nu(B), \forall A \subseteq \cX, B \subseteq \cY
		  \right\}.
	  \end{equation}
  \end{definition}

  \begin{definition}
	  Para una función medible $T:\cX \to \cY$ se define el \emph{operador push-forward} de $T$ como la aplicación $\Tpf:\ProbSpace[\cX] \to \ProbSpace[\cY]$ que satisface la siguiente relación:
	  \begin{equation}
		  \label{eq:pushForward}
		  \int_{\cX} f(x) \dd{\Tpf\mu(x)} = \int_{\cY} f(T(x)) \dmu[x], \quad \forall f \in \mathcal{C}(\cY),
	  \end{equation}
	  para toda $\mu \in \ProbSpace[\cX]$. Adicionalmente, el operador push-forward se puede definir como aquel operador que satisface la siguiente relación:
	  \begin{equation}
		  \label{eq:pushForward2}
		  \forall A \subseteq \cY \text{ medible}, \quad \Tpf\mu(A) = \mu(T^{-1}(A)).
	  \end{equation}
  \end{definition}

  \begin{remark}
	  Se puede notar que $\Tpf$ preserva la positividad y la masa total, es decir, si $\mu \in \ProbSpace[\cX]$, entonces $\Tpf\mu \in \ProbSpace[\cY]$.
  \end{remark}

  \begin{remark}
	  Para el caso en que la medida $\mu \in \ProbSpace$ sea una medida discreta\footnote{i.e. $\mu = \sum_{i=0}^{n} m_i \delta_{x_i}$ con $m\in\Simplex$, $x_1,\ldots, x_n\in\cX$ y $\delta_x$ la medida de Dirac en $x$}, entonces el operador $\Tpf$ lo que hará será intercambiar la masa de cada punto de $\cX$ a su imagen en $\cY$, es decir,
	  \begin{equation}
		  \label{eq:pushForwardDiscreto}
		  \Tpf\mu = \sum_{i=0}^{n} m_i \delta_{T(x_i)}.
	  \end{equation}

  \end{remark}



 }

\section{El problema de transporte}
 {

  \subsection*{El problema de Monge}
  {
	  En esta sección se introducirá el problema de transporte óptimo de Monge. En este problema se busca transportar la masa de la medida $\mu \in \ProbSpace[\cX]$ a la medida $\nu \in \ProbSpace[\cY]$ a través de una función medible $T: \cX \to \cY$, a la que llamaremos como \emph{función de transporte} o \emph{mapa de transporte}, utilizando una \emph{función de coste} $c(x, y)$, que representa el costo de transportar la masa del punto $x$ al punto $y$. Más formalmente, este problema se puede definir de la siguiente manera:

	  \begin{definition}
		  Dados $\mu\in \ProbSpace[\cX]$ y $\nu \in \ProbSpace[\cY]$, se define el \emph{problema de transporte óptimo} como el problema de encontrar una función de transporte $T: \cX \to \cY$ que minimice el costo total de transporte, es decir, que minimice la siguiente expresión:
		  \begin{equation}
			  \label{eq:problemaTransporte}
			  \inf_{T: \Tpf \mu = \nu} \int_{\cX} c(x, T(x)) \dmu[x],
		  \end{equation}
	  \end{definition}

	  \begin{remark}
		  \label{remark:problemaTransporteMongeDiscreto}
		  Cuando las medidas $\mu$ y $\nu$ son discretas, es decir, se representan de la siguiente manera:
		  \begin{align}
			  \mu & = \sum_{i=0}^{n} m_i \delta_{x_i}, &
			  \nu & = \sum_{j=0}^{m} n_j \delta_{y_j},
		  \end{align}
		  donde $m \in \Simplex[n]$, $n \in \Simplex[m]$, $x_i \in \cX$, $y_j \in \cY$, entonces el problema de transporte óptimo se puede representar de la siguiente manera:
		  \begin{equation}
			  \label{eq:problemaTransporteDiscreto}
			  \inf_{T: T(x_i) = y_j} \sum_{i=0}^{n} c(x_i, T(x_i)) m_i.
		  \end{equation}

		  Cabe destacar que este problema no siempre tiene solución (generalmente no la tiene si $m > n$), y que en caso de tenerla, no siempre es única.\FM[inline]{Se podría agregar un ejemplo, el de los cuatro puntos en el plano.}
	  \end{remark}
  }

  \subsection*{El problema de Kantorovich}
  {
	  Como se mencionó en la Observación \ref*{remark:problemaTransporteMongeDiscreto}, el problema de transporte óptimo de Monge no siempre tiene solución, y en caso de tenerla, no siempre es única. Motivado por esto, Kantorovich propuso una formulación alternativa del problema de transporte óptimo, que si tiene solución (aunque puede que no sea única). Este problema se puede definir de la siguiente manera:
	  \begin{definition}
		  Dados $\mu \in \ProbSpace[\cX]$ y $\nu \in \ProbSpace[\cY]$, se define el \emph{problema de transporte óptimo de Kantorovich} como el problema de encontrar una medida de probabilidad $\gamma \in \Cpl(\mu, \nu)$ que minimice el costo total de transporte, es decir, que minimice la siguiente expresión:
		  \begin{equation}
			  \label{eq:problemaTransporteKantorovich}
			  \inf_{\gamma \in \Cpl(\mu, \nu)} \int_{\cX \times \cY} c(x, y) \dgamma[x, y] .
		  \end{equation}
		  Al conjunto de medidas de probabilidad que resuelven este problema se le llama \emph{transportes óptimos}.\FM[inline]{Se podría agregar el típico dibujo en el que se muestra el copling}
	  \end{definition}
	  \FM[inline]{Tenía pensado en poner el teorema de Brenier (teo 2.1 de \cite{peyre2019computational}), pág 27 para explicar la equivalencia Kantorovich-Monge.}
	  \FM[inline]{Incluir la formulación de Kantorovich discreto?}
	  \FM[inline]{Mencionar que esta formulación es más general, y que permite resolver problemas en los que no se puede definir una función de transporte?}
	  \FM[inline]{Incluir en este punto el teorema de dualidad de Kantorovich-Rubinstein?}

  }

  \section{La Distancia y el Espacio de Wasserstein}\label{sec:la-distancia-y-el-espacio-de-Wasserstein}
  {
	  \FM{Incluir subsections para la distancia, espacio, y convergencia débil?}
	  En esta sección se demostrará que, al evaluar la expresión \eqref{eq:problemaTransporteKantorovich} para una función de coste con distancia, se obtiene una distancia entre medidas de probabilidad. Revisaremos algunas propiedades de esta distancia, para concluir que esta distancia metriza la convergencia débil entre medidas de probabilidad.

	  \begin{definition}[La distancia de Wasserstein]\label{def:distanciaWasserstein}
		  Sea $(\cX, d)$ un espacio Polaco y sea $p \geq 1$. Para dos medidas $\mu, \nu$ sobre $\cX$, la distancia de Wasserstein de orden $p$ entre $\mu$ y $\nu$ es definida por medio de la fórmula
		  \begin{equation}
			  \label{eq:distanciaWasserstein}
			  \Wasserstein[p]{\mu}{\nu}  \eqdef \left( \inf_{\gamma \in \Cpl(\mu, \nu)} \int_{\cX \times \cX} d(x, y)^{p} \dgamma[x, y] \right)^{\frac{1}{p}}.
		  \end{equation}

	  \end{definition}

	  \begin{example}
		  $\Wasserstein[1]{\delta_x}{\delta_y} = d(x, y)$. Notemos que en este ejemplo, se puede interpretar que la distancia de Wasserstein metriza el ``esfuerzo'' de llevar la masa del punto $x$ al punto $y$.
	  \end{example}

	  Notemos que, en estricto rigor, $W_p$ no es una distancia en sí, dado que puede tomar valores de $+\infty$, sin embargo, se puede demostrar que $W_p$ satisface los axiomas de ser una distancia. No se demostrará este hecho, pero se puede encontrar una demostración en \cite{villani2009optimal}, pág 94.

	  Por tanto, resulta natural definir el espacio en el que la distancia de Wasserstein tome valores finitos.

	  \begin{definition}[El espacio de Wasserstein]
		  Con los mismos supuestos que en la Definición \ref{def:distanciaWasserstein}, se define el espacio de Wasserstein de orden $p$ por medio de
		  \begin{equation}
			  \WassersteinSpace[p]{\cX} \eqdef \left\{
			  \mu \in \ProbSpace[\cX] \colon \int_{\cX} d(x, x_0)^{p} \dmu[x] < \infty
			  \right\},
		  \end{equation}
		  donde $x_0 \in \cX$ es un punto fijo arbitrario. De esta forma, $W_p$ define una distancia (finita) sobre $\WassersteinSpace[p]{\cX}$.
	  \end{definition}
	  En palabras simples, el espacio de Wasserstein de orden $p$ es el conjunto de medidas de probabilidad en $\cX$ cuyo momento de orden $p$ es finito. Lo interesante del espacio de Wasserstein, es que su respectiva distancia lo metriza, como lo dice el siguiente teorema:
	  \begin{theorem}\label{thm:espacioWassersteinEsMetrico}
		  Si $(\cX, d)$ es un espacio Polaco, entonces el espacio de Wasserstein $\WassersteinSpace[p]{\cX} $, metrizado por la distancia de Wasserstein $W_p$, es también un espacio Polaco.
	  \end{theorem}

	  \begin{proof}
		  Revisar la demostración del Teorema 6.18 en \cite[p. 105]{villani2009optimal}
	  \end{proof}

	  A partir de ahora, se asumirá que el espacio $\WassersteinSpace[p]{\cX} $ siempre estará equipado con su respectiva distancia $W_p$.

	  \begin{remark}
		  A través de la desigualdad de Hölder, se puede demostrar que para $p \leq q$, se tiene que $\Wasserstein[p]{\mu}{\nu} \leq \Wasserstein[q]{\mu}{\nu}$, para toda $\mu, \nu \in \WassersteinSpace[p]{\cX}$. Y por tanto, las topologías inducidas por las distancias de Wasserstein se van encajonando.

		  En particular, la distancia de Wasserstein de orden 1, es la más débil de todas. Como norma general, la distancia $W_1$  es la más flexible y fácil de acotar, mientras que la distancia $W_2$ posee mejores propiedades geométricas, pero es más difícil de trabajar.
	  \end{remark}

	  Vista la distancia y el espacio de Wasserstein, se presentará una caracterización de convergencia en este espacio. Para ello, se definirá la convergencia débil entre medidas de probabilidad.

	  \begin{definition}[Convergencia Débil]
		  Sea $(\cX, d)$ un espacio Polaco y sea $p \geq 1$. Se dice que una sucesión de medidas de probabilidad $(\mu_n)_{n \in \N} \subset \WassersteinSpace[p]{\cX} $ converge débilmente a $\mu \in \WassersteinSpace[p]{\cX}$ si
		  \begin{equation}
			  \forall \phi \in \ContBoundedSpace[\cX], \quad \int_{\cX} \phi(x) \dd{\mu_n(x)} \to \int_{\cX} \phi(x) \dmu[x].
		  \end{equation}
		  y lo denotaremos por $\mu_n \wto \mu$.
	  \end{definition}

	  \begin{note}
		  Intuitivamente, que una sucesión de medidas de probabilidad converjan débilmente a una medida $\mu$ significa que es la forma ``más fácil'' que tiene la sucesión de converger a $\mu$.
	  \end{note}

	  \begin{theorem}[La Distancia de Wasserstein Metriza la Convergencia Débil]
		  Sea $(\cX, d)$ un espacio Polaco y sea $p \geq 1$. Entonces, la distancia de Wasserstein $W_p$  metriza la convergencia débil en $\WassersteinSpace[p]{\cX}$.
	  \end{theorem}

	  \begin{remark}
		  En otras palabras, si $(\mu_n)_{n\in\N}$ es una sucesión de medidas de probabilidad en $\WassersteinSpace[p]{\cX}$ y $\mu\in \WassersteinSpace[p]{\cX} $ otra medida, entonces $\mu_n \wto \mu$ si y sólo si $\Wasserstein[p]{\mu_n}{\mu} \to 0$.
	  \end{remark}

	  \begin{example}
		  Consideremos las siguientes distancias y divergencias entre medidas de probabilidad:
		  \begin{gather*}
			  \TV{\mu}{\nu} \eqdef \sup_{A \subseteq \cX} \abs{\mu(A) - \nu(A)}, \\
			  \KL{\mu}{\nu} \eqdef \int_{\cX} \log\left(\dv{\mu}{\nu}(x)\right) \dmu[x], \\
			  \JS{\mu}{\nu} \eqdef \KL{\mu}{\frac{\mu + \nu}{2}} + \KL{\nu}{\frac{\mu + \nu}{2}},
		  \end{gather*}
		  donde la primera es la distancia total variación, la segunda es la divergencia de Kullback-Leibler, y la tercera es la divergencia de Jensen-Shannon.

		  Si consideramos $\delta_\theta$ y $\delta_0$ medidas de Dirac centradas en $\theta$ y $0$ respectivamente, entonces se puede demostrar que
		  \begin{align*}
			  \Wasserstein[1]{\delta_\theta}{\delta_0} & = |\theta|                            &
			  \TV{\delta_\theta}{\delta_0}             & = \begin{cases}
				                                               1 & \text{si } \theta \neq 0 \\
				                                               0 & \text{si } \theta = 0
			                                               \end{cases}          \\
			  \KL{\delta_\theta}{\delta_0}             & = \begin{cases}
				                                               +\infty & \text{si } \theta \neq 0 \\
				                                               0       & \text{si } \theta = 0
			                                               \end{cases} &
			  \JS{\delta_\theta}{\delta_0}             & = \begin{cases}
				                                               \log(2) & \text{si } \theta \neq 0 \\
				                                               0       & \text{si } \theta = 0
			                                               \end{cases}
		  \end{align*}
		  Entonces, si tomamos $\theta = \frac{1}{n} $ y dejamos que $n \to \infty$, se tiene que $\Wasserstein[1]{\delta_\theta}{\delta_0} \to 0$, pero el resto de distancias y divergencias no convergen a 0.
		  Por tanto, se puede notar que la distancia de Wasserstein es la única que es capaz de distinguir entre medidas de probabilidad que no tienen soporte en el mismo punto, gracias a que metriza la convergencia débil.
	  \end{example}

  }  % end of sec. La Distancia y el Espacio de Wasserstein

  \section{El Baricentro de Wasserstein Bayesiano}\label{sec:el-baricentro-de-Wasserstein-Bayesiano}
  {
	  \subsection*{La Media de Fréchet}\label{ssec:la-media-de-Frechet}
	  {
		  En esta sección se revisará el concepto de media de Fréchet, el cual es una generalización de la noción de promedio para espacios métricos. Este concepto será clave para definir el baricentro de Wasserstein.
		  \begin{definition}[Funcional y Media de Fréchet]
			  Sea $(\cX, d)$ un espacio Polaco. Sean $x_1,$ $\ldots, x_n$ puntos en $\cX$ y sean $w_1,\ldots,w_n\in\R$ pesos asociados a los puntos. Para cada $p\in\cX$, se define el \emph{funcional de Fréchet} por
			  \begin{equation}
				  \label{eq:funcionalFrechet}
				  \Psi(p) \eqdef \sum_{i=1}^{n} w_i d(p, x_i)^2.
			  \end{equation}
			  Y, en caso de que exista un punto $m \in \cX$ que minimice el funcional $\Psi$, entonces este se definirá como la \emph{media de Fréchet} de los puntos $x_1,\ldots, x_n$. Es decir, es aquel punto tal que minimiza el siguiente problema:
			  \begin{equation}
				  \label{eq:mediaFrechet}
				  m \eqdef \argmin_{p \in \cX} \sum_{i=0}^{n} w_i d(p, x_i)^2.
			  \end{equation}
		  \end{definition}

		  \begin{example}\label{ex:baricentro-triangulo}
			  Tomemos $x_1, x_2, x_3 \in \R^2$ tres puntos en el plano, formando un triángulo. Si se define el promedio (o el \textit{baricentro}, en el contexto de un triángulo) de estos puntos por $\bar x = \frac{1}{3} (x_1 + x_2 + x_3)$, entonces se puede comprobar fácilmente que este es el único que minimiza el funcional de Fréchet:
			  \begin{equation}
				  F(p) = \frac{1}{3} \sum_{i=0}^{3} \|p - x_i\|^2.
			  \end{equation}
			  Dado que este funcional se puede descomponer de la siguiente manera:
			  \begin{equation}
				  F(p) = F(\bar x) + \|p-\bar x\|^2,
			  \end{equation}
			  se puede ver que la media de Fréchet generaliza la noción de promedio.
		  \end{example}

		  \begin{remark}
			  La razón por la que resulta interesante estudiar este concepto, es que sólo utiliza nociones métricas, y se desliga de la noción vectorial. Como se vió en el Ejemplo \ref{ex:baricentro-triangulo}, el promedio utilizó nociones vectoriales (suma, ponderación) mientras que la media de Fréchet utilizó nociones métricas, resultando en el mismo promedio.

			  Sin embargo, el hecho de que se pueda cambiar la distancia, hace que el baricentro cambie, y dependa de ésta.
		  \end{remark}



	  }  % end of sec. La Media de Fréchet

	  \subsection*{El Baricentro de Wasserstein}\label{ssec:el-baricentro-de-Wasserstein}
	  {
		  Como se vió en la sección anterior, la media de Fréchet permite definir una noción de promedio, en espacios métricos. El Teorema \ref{thm:espacioWassersteinEsMetrico} nos dice que $(\WassersteinSpace[p]{\cX}, W_p)$ es un espacio métrico, y por tanto, se puede definir su respectivo ``promedio'':
		  \begin{definition}
			  Sean $\mu_1,\ldots, \mu_n \in \WassersteinSpace[p]{\cX} $ y sean $w = (w_1,\ldots, w_n) \in \Simplex[n]$ sus pesos asociados. El \emph{baricentro de Wasserstein} se define por medio de
			  \begin{equation}
				  \bar \mu \eqdef \arginf_{\nu \in \WassersteinSpace[p]{\cX} } \sum_{i=0}^{n} w_i \Wasserstein[p]{\nu}{\mu_i}^p
			  \end{equation}

		  \end{definition}
		  \FM[inline]{Se podría poner algún ejemplo con una imagen.}

		  Es posible generalizar aún más la noción de baricentro de Wasserstein a una colección infinita de medidas. Esto se puede hacer considerando una medida $\Gamma \in \ProbSpace[\ProbSpace ] $, que cumplirá el rol de los pesos $w_1,\ldots, w_n $ en la definición anterior. Esto se puede formalizar en la siguiente definición:

		  \begin{definition}
			  Sea $\Gamma \in \ProbSpace[\ProbSpace]$ una medida. El baricentro de Wasserstein se puede (re)-definir como aquel que minimice el siguiente problema:
			  \begin{equation}
				  \bar \mu \eqdef \arginf_{\mu \in \ProbSpace} \int_{\ProbSpace} \Wasserstein[p]{\mu}{\nu}^p \dd{\Gamma(\nu)}
			  \end{equation}

		  \end{definition}


	  }  % end of sec. El Baricentro de Wasserstein


	  \FM[inline]{Mencionar las geodésicas en el espacio de Wasserstein}

	  \FM[inline]{Baricentro de Wasserstein Bayesiano}
  }  % end of sec. El Baricentro de Wasserstein
 }