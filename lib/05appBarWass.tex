\chapter{Aplicaciones a los Baricentros de Wasserstein}

\RED[inline]{Hay que revisar esta introducción al capítulo en la reu del 02/07}
En este capítulo se presenta el tema principal de la tesis: la implementación del Descenso del Gradiente Estocástico en el Espacio de Wasserstein, y su aplicación a la estimación del Baricentro de Wasserstein Bayesiano.

Para ello, se empieza explicando la implementación del SGDW, y se realizan experimentos con alguos muestreadores de distribuciones. Luego, se propone una adaptación del SGDW, donde se proyecta el baricentro sobre alguna variedad $\Manifold$ de medidas para que se vea más natural y atractivo. Finalmente, se utilizan estas implementaciones para el cálculo del BWB, donde se propone una técnica para poder estimarla utilizando una GAN como prior.

\section{Descenso del Gradiente Estocástico sobre el Espacio de Wasserstein}\label{sec:sgdw}  % MARK: SGDW

\FM[inline]{Podría ser de título ``Implementación del SGDW''? ya lo definí anteriormente}

En esta sección se presenta la implementación del SGDW.

\subsection{Interpretación de una Imagen como Medida}\label{ssec:interpr-imagen-medida}  % MARK: - Interpretación de una Imagen como Medida

Recordar que si $\mu\in \ProbSpace[\cX] $ es una medida discreta, entonces esta queda definida de la siguiente forma:
\begin{equation}\label{eq:medida-discreta}
    \mu = \sum_{i=1}^{n} m_i \delta_{x_i},
\end{equation}
donde $m \in \Simplex[n]$ es un vector de probabilidad y $ \left\{ x_1, \dots, x_n \right\} \subseteq \cX $ son sus posiciones. En el caso de una imagen, se puede interpretar como una medida discreta, donde $x_i$ es el $i$-ésimo píxel y $m_i$ es la intensidad de la imagen en el $i$-ésimo píxel. Por lo tanto, se puede muestrear un píxel de una imagen de la misma forma que se muestrea una medida discreta.

\FM[inline]{Es posible que este párrafo este de más.}

Cabe destacar que por definición, realizar un muestreo $x \sim \mu$ es equivalente a muestrear un índice $i \sim \Categorical(m)$ y retornar $x = x_i$. Por este motivo, se implementan las medidas discretas siendo extendidas de la clase \texttt{Categorical} del módulo \texttt{torch.distributions} de \textit{PyTorch}. Del mismo modo, como una imagen es una medida discreta, se extiende la clase para una medida discreta, tal que sea eficiente en memoria y en tiempo de construcción.

\subsection{Implementación del Algoritmo}\label{ssec:implementacion-algoritmo}  % MARK: - Implementación del Algoritmo

Dado que el Algoritmo~\ref{alg:sgdw-clasico} está diseñado para medidas absolutamente continuas, se reinterpreta este algoritmo para que se pueda trabajar con medidas discretas. Para ello, se destaca que la Definición~\ref{def:sgdw} se puede reinterpretar como la $\eta_k$-interpolación geodésica entre las medidas $\mu_k$ y $\tilde \mu_k$, mientras que la Definición~\ref{def:bsgdw} se puede reinterpretar como el baricentro de las medidas $\qty( \mu_k, \tilde\mu_k^{(1)}, \dots, \tilde\mu_k^{(S_k)} )$ con pesos $\qty(1-\eta_k, \frac{\eta_k}{S_k}, \dots, \frac{\eta_k}{S_k}) \in \Simplex[S_k+1]$. De este modo, el Algoritmo~\ref{alg:sgdw-clasico} se extiende de la siguiente manera:
\begin{algorithm}[H]
    \caption{SGDW General}
    \label{alg:sgdw-general}
    \begin{algorithmic}[1]
        \Require Acceso a las muestras de $\Gamma(\dd \mu) \in \ProbSpace[\ProbSpace]$, un esquema de paso $(\eta_k)_k \in [0, 1]^\N$ y un esquema de paso $(S_k)_k \in \N^\N$.
        \State{$k\gets0$}
        \State{Muestrear $\mu_0 \sim \Gamma$}
        \Repeat
        \State{Muestrear $\tilde \mu_k^{(1)}, \dots, \tilde \mu_k^{(S_k)} \simiid \Gamma$}
        \State{$\gamma\gets\qty(1-\eta_k, \frac{\eta_k}{S_k}, \dots, \frac{\eta_k}{S_k})$}
        \State Definir $\mu_k$ como el baricentro de $\qty( \mu_k, \tilde\mu_k^{(1)}, \dots, \tilde\mu_k^{(S_k)} )$ con pesos $\gamma$.
        \State{$k\gets k+1$}
        \Until{un criterio de detención ha sido alcanzado.}
        \State\Return $\mu_k$
    \end{algorithmic}
\end{algorithm}

Como se está trabajando con imágenes, se puede aprovechar su estructurar para calcular una estimación de los baricentros de manera más eficiente, utilizando el algoritmos de Baricentros de Wasserstein Convolucionales \cite{solomon2015convolutional} o su versión Insesgada (\textit{Debiased} en inglés) \cite{janati2020debiased}. Sin embargo, el Algoritmo~\ref{alg:sgdw-general} es lo suficientemente general para ser aplicado a cualquier medida discreta, utilizando la estimación del algoritmo de Sinkhorn \cite{cuturi2013sinkhorn}, por ejemplo. Todos estos métodos de cálculo de baricentros se implementan de manera eficiente utilizando la librería de \textit{Python Optimal Transport} (POT) \cite{flamary2021pot}, donde además esta librería admite la paralelización de los cálculos por medio del GPGPU \cite{owens2008gpu}.

\FM[inline]{Mejorar esto utilizando la medida $\Prob_X$}

Para muestrear a partir de una medida $\Gamma$ a partir de un conjunto de datos $\left\{ \mu_i \right\}_{i=1}^{N} \subseteq \ProbSpace[\cX] $, se puede calcular la medida empírica $\hat \Gamma$ de la siguiente forma:
\begin{equation}\label{eq:medida-empirica}
    \hat \Gamma (\dd \mu) = \frac{1}{N} \sum_{i=1}^{N} \delta_{\mu_i} (\dd \mu).
\end{equation}
Esto corresponde muestrear una imagen cualquiera de forma equiprobable. Otra manera de obtener una $\Gamma$, es a través de un modelo generativo, de manera que muestrear una imagen correspondería a simplemente muestrear un ruido aleatorio $z \sim \Prob_Z$ y aplicar la función generadora $G_\theta(z)$.

\subsection{Resultados y Discusión}\label{ssec:sgdw-resultados-discusion}  % MARK: -- Resultados y Discusión

\FM[inline]{En esta sección se podría incluir el cálculo de un baricentro, tanto del dataset como de la GAN}

\subsection{Conclusiones}\label{ssec:sgdw-conclusiones}  % MARK: -- Conclusiones

\FM[inline]{Insertar aquí alguna conclusión}

\FM[inline]{Una conclusión que se me ocurre es como los dos baricentros, el del conjunto de datos y el de la GAN se parecen, algo que debería de pasar puesto que ambos están aproximando a alguna medida de referencia $\Prob_X$}


% \newpage
\section{SGDW Proyectado}\label{sec:sgdwp}  % MARK: - Section SGDWP


A pesar de que la implementación del SGDW entrega los resultados esperados, el baricentro no parece ser lo suficientemente natural. Por este motivo, se propone una adaptación del SGDW, donde se proyecta el baricentro sobre alguna variedad $\Manifold$ de medidas para que se vea más natural y atractivo.

Para ello, se recapitula la definición de CWB de la Sección~\ref{sec:app-bar-wass-Proyectados}, donde en este trabajo \cite{simon2020barycenters} se explica la forma de proyectar una medida $\mu$ sobre una variedad específica de medidas $\Manifold$, aprendidas por una red generativa.

En este sentido, el conjunto de modelos $\Manifold\subseteq \ProbSpace[\cX] $ correspondería a la variedad generada por la red $G_\theta$, es decir,
\begin{equation}
    \Manifold \eqdef \left\{ G_\theta(z) \in \ProbSpace[\cX] \colon z \in \cZ \right\} \subseteq \ProbSpace[\cX].
\end{equation}
Este cambio puede hacer que el Supuesto~\ref{assump:caso-particular-geodesicamente-convexo} no se cumpla, dado que es posible que $\Manifold$ no sea geodésicamente convexo. Esto provocaría que el Teorema~\ref{thm:convergencia-sgdw} y la Proposición~\ref{prop:convergencia-bsgdw} no garanticen la existencia o la unicidad del baricentro proyectado.

awa


\FM[inline]{Para comprobar estas hipótesis, se realizan los siguientes experimentos.....}
\FM[inline]{Aquí quizás incluir que, por este motivo, se harán experimentos para ver si converge a una única medida o no.}


\subsection{Integración de la Proyección al SGDW}\label{ssec:sgdwp-deduccion-algoritmo}  % MARK: - Integración de la Proyección al SGDW

A pesar de que los autores de \cite{simon2020barycenters} explican que, para obtener los resultados de su artículo utilizan el Algoritmo~\ref{alg:ADMM-CWB}, la realidad es que al revisar su código fuente \cite{imagebar2020simon} se observa que sólo utilizan una iteración del algoritmo anterior (es decir, no utilizan el Lagrangiano Aumentado). De este modo, el algoritmo se simplifica de la siguiente manera: para dos medidas $\mu_0, \mu_1 \in \ProbSpace[\cX] $ y un número $t \in [0, 1]$, empiezan por calcular la $t$-interpolación geodésica de estas medidas, y luego proyectan este baricentro sobre la variedad $\Manifold$ a través del AE.

Motivados por esta simplificación, se propone una adaptación del Algoritmo~\ref{alg:sgdw-general} para que proyecte el baricentro en la variedad $\Manifold$, donde además se agrega el parámetro $n_P\in\N$ para proyectar cada $n_P$ iteraciones del algoritmo. De este modo, se deduce el Algoritmo~\ref{alg:sgdwp} para el SGDW Proyectado.
\begin{algorithm}[H]
    \caption{SGDW Proyectado (SGDWP)}
    \label{alg:sgdwp}
    \begin{algorithmic}[1]
        \Require Acceso a las muestras de $\Gamma(\dd \mu) \in \ProbSpace[\ProbSpace]$, un esquema de paso $(\eta_k)_k \in [0, 1]^\N$, un esquema de paso $(S_k)_k \in \N^\N$, un proyector $P:\ProbSpace[\cX] \to \Manifold\subseteq \ProbSpace[\cX] $ y un número $n_P \in \N$.
        \State{$k\gets0$}
        \State{Muestrear $\mu_0 \sim \Gamma$}
        \Repeat
        \State{Muestrear $\tilde \mu_k^{(1)}, \dots, \tilde \mu_k^{(S_k)} \simiid \Gamma$}
        \State{$\gamma\gets\qty(1-\eta_k, \frac{\eta_k}{S_k}, \dots, \frac{\eta_k}{S_k})$}
        \State Definir $\mu_k$ como el baricentro de $\qty( \mu_k, \tilde\mu_k^{(1)}, \dots, \tilde\mu_k^{(S_k)} )$ con pesos $\gamma$.
        \If{$k \mod n_P = 0$}
        \State{$\mu_k\gets P(\mu_k)$}\Comment{Proyectar $\mu_k$ sobre $\Manifold$}
        \EndIf
        \State{$k\gets k+1$}
        \Until{un criterio de detención ha sido alcanzado.}
        \State{$\mu_k\gets P(\mu_k)$}\Comment{Terminar con una última proyección antes de retornar.}
        \State\Return $\mu_k$
    \end{algorithmic}
\end{algorithm}

\FM[inline]{Este párrafo podría ser parte de las conclusiones? Onda, por el trabajo futuro.}

Cabe destacar que el Algoritmo~\ref{alg:sgdwp} tiene una mecánica similar al Método del Gradiente Proyectado (MPG)~\cite[Secc. 5.1]{optimizacion2022amaya}, puesto que este es un método de optimización con restricciones que, si en una iteración se viola alguna restricción, entonces se proyecta dicha iteración sobre el conjunto de restricciones. Este es el caso del Algoritmo~\ref{alg:sgdwp}, donde se proyecta el baricentro en la variedad $\Manifold$. Sin embargo, se diferencia del MPG, en que el dominio $\Manifold$ puede no ser (geodésicamente) convexo.

\FM[inline]{debería mencionar como se implemnetó el proyector $P$ en más detalle?}


\subsection{Resultados y Discusión}\label{ssec:sgdwp-resultados-discusion}  % MARK: - Resultados y Discusión

\subsection{Conclusiones}\label{ssec:sgdwp-conclusiones}  % MARK: - Conclusiones


% \newpage
\section{Baricentro de Wasserstein Bayesiano}\label{sec:bwb}  % MARK: - Section Baricentro de Wasserstein Bayesiano


Teniendo un conjunto de modelos finito, dígase, $\Models \subseteq \ProbSpace[\cX] $ con $| \Models | = N < + \infty$, un acercamiento ``ingenuo'' para calcular la medida posterior sobre este espacio debría considerar el vector de verosimilitudes $L_n = (\cL_n(\mu))_{\mu \in \Models} \in \R^N$. El problema con este enfoque es que, por definición, es bastante probable que uno de los puntos en que se evalúa $\rho_\mu$ sea $0$. Es decir, si $\exists x_i \in D \colon \rho_\mu(x_i) = 0$, entonces ese modelo tendrá una verosimilitud nula. En la práctica, esto sucede con mucha frecuencia. Por este motivo, se decide tomar otro enfoque.


\subsection{Construcción de la Posterior Usando una GAN}\label{ssec:construccion-posterior}  % MARK: - Construcción de la Posterior Usando una GAN

\RED[inline]{Hay que seguir revisando esta sección}

Como se explicó en secciones anteriores,\FM{Aquí sería bueno incluir las secciones de lo que se habla esto, sguramente el de la GAN y WGAN}
dada una medida de referencia $\Prob_X$\footnote{del cuál se tiene acceso a través de una muestra para obtener una medida empírica $\hat\Prob_X = \frac{1}{N}\sum_{i=1}^{N} \delta_{x_i}$}\FM{Corregir} lo que hacen las redes generativas es aproximarla por medio de un modelo generativo $\Prob_G$. Gracias a esta propiedad, se propone utilizar una GAN como prior para poder calcular la posterior. Cabe destacar que, a pesar de que la idea de utilizar una GAN como prior fue original, ya existía un trabajo que propone algo similar \cite{patel2019bayesian}, sin embargo, en este trabajo de tesis se formalizan estas ideas.

Dada una red generadora $G_\theta \colon \cZ \to \ProbSpace[\cX] $ con una medida en el espacio latente $\Prob_Z$, se propone utilizar como prior a la medida
\begin{equation}
    \Pi^G \eqdef \pf{G_\theta} \Prob_Z.
\end{equation}
De esta manera, la posterior tendría la siguiente forma:
\begin{equation}
    \Pi_n(\dd \mu)
    \eqdef \frac{\cL_n(\mu)}{\int_{\ProbSpace[\cX]} \cL_n(\nu) \; \Pi^G(\dd \nu)} \; \Pi^G(\dd \mu)
    = C^{-1} \cL_n(\mu) \; \Pi^G(\dd \mu),
\end{equation}
donde $C \eqdef \int_{\ProbSpace[\cX]} \cL_n(\nu) \; \Pi^G(\dd \nu)$ es una constante de normalización.

Dada alguna función arbitraria $\Pi_n$-integrable $g$ (como por ejemplo $\mu \mapsto \Wasserstein[p]{\mu}{\nu}^p$), se puede comprobar lo siguiente:
\begin{align}
     & \int_{\ProbSpace[\cX]} g(\mu) \; \Pi_n(\dd \mu)                         \\
     & = C^{-1} \int_{\ProbSpace[\cX]} g(\mu) \cL_n(\mu) \; \Pi^G(\dd \mu)     \\
     & = C^{-1} \int_{\cZ} g(G_\theta(z)) \cL_n(G_\theta(z)) \; \Prob_Z(\dd z) \\
     & = \int_{\cZ} g(G_\theta(z)) \; \Pi^Z_n(\dd z),
\end{align}
donde $\Pi^Z_n(\dd z)$ es la \textit{medida posterior en el espacio latente} y se define por
\begin{equation}
    \Pi^Z_n(\dd z) \eqdef C^{-1} \cL_n(G_\theta(z)) \; \Prob_Z(\dd z).
\end{equation}

Por la Definición~\ref{def:operador-push-forward} del operador push-forward, se concluye la siguiente propiedad de la medida posterior:
\begin{equation}\label{eq:posterior-en-latente}
    \Pi_n = \pf{G_\theta} \Pi^Z_n.
\end{equation}
La forma de interpretar la ecuación~\eqref{eq:posterior-en-latente} es que para obtener un muestreo de la posterior $\Pi_n(\dd \mu)$ basta con hacer un muestreo de la posterior en el espacio latente $\Pi^Z_n(\dd z)$ y después aplicar la red generadora $G_\theta$.

Esto resulta beneficioso, pues delega la tarea de muestrear a partir de la posterior al espacio latente $\cZ$ (el cuál, en la práctica, es $\R^{d_z}$), la cual es mucho más fácil de simular. Por ejemplo, una manera de obtener muestreos a partir de $\Pi^Z_n$, es por medio del método de Markov Chain Monte Carlo (MCMC) \cite{andrieu2003introduction,brooks2011handbook,goodman2010ensemble}.\RED{Revisar este párrafo}






% Esto es, dado una red generadora $G_\theta \colon \cZ \to \ProbSpace[\cX] $, se propone como prior sobre el conjunto de modelos de la siguiente manera:
% \begin{equation}
%     \Pi_G(\dd \mu) \eqdef \pf{G_\theta} \Prob_Z(\dd \mu).
% \end{equation}
% De manera que la posterior tendría la siguiente forma:
% \begin{equation}
%     \Pi_n(\dd \mu)
%     \eqdef \frac{\cL_n(\mu)}{\int_{\ProbSpace[\cX]} \cL_n(\nu) \Pi_G(\dd \nu)} \Pi_G(\dd \mu)
%     = \frac{1}{C} \cL_n(\mu) \Pi_G(\dd \mu),
% \end{equation}
% donde $C \eqdef \int_{\ProbSpace[\cX]} \cL_n(\nu) \Pi_G(\dd \nu)$ es una constante de normalización.


% De este modo, la función valor del problema de optimización de la Definición~\ref{def:baricentroWassersteinBayesiano} se puede reescribir de la siguiente manera:
% \begin{align}
%      & \int_{\ProbSpace[\cX]} \Wasserstein[p]{\mu}{\nu}^p \; \Pi_n(\dd \nu)                              \\
%      & = \frac{1}{C} \int_{\ProbSpace[\cX]} \Wasserstein[p]{\mu}{\nu}^p \cL_n(\nu) \; \Pi_G(\dd \mu)     \\
%      & = \frac{1}{C} \int_{\cZ} \Wasserstein[p]{\mu}{G_\theta(z)}^p \cL_n(G_\theta(z)) \; \Prob_Z(\dd z)
% \end{align}

% Con este enfoque, notamos que la función de probabilidad de densidad de la medida posterior en el espacio latente es:
% \begin{equation}
%     \Pi_n^Z (\dz) \eqdef \frac{1}{C} \cL_n(G_\theta(z)) \; \Prob_Z(\dz).
% \end{equation}
% Y de este modo, la medida posterior en el espacio de los modelos es:

% El objetivo de obtener esta función de distribución, es que nos dice que podemos simular muestras de la posterior $\Pi_n(\dd \mu)$ a través de la red generativa $G_\theta$. En efecto, basta darse cuenta que






\subsection{Resultados y Discusión}\label{ssec:bwb-resultados-discusion}  % MARK: - Resultados y Discusión

\subsection{Conclusiones}\label{ssec:bwb-conclusiones}  % MARK: - Conclusiones

