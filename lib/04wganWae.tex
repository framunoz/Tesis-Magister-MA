\chapter{Unificando la WGAN y el WAE}\label{chap:WAE-WGAN}  % MARK: Unificando la WGAN y el WAE

En capítulos anteriores se explica que tanto las WGANs como los WAEs aproximan la distribución de alguna distribución real $\Prob_X$, además de que permiten aproximar a variedades, y proyectar sobre ellas.
Por esta razón, en este trabajo se estima una distribución de imágenes $\Prob_X$ para tener acceso a un muestreo rápido de este, además, para que los baricentros sean naturales, se utiliza una estructura de AE para proyectar sobre la variedad de imágenes, siguiendo el trabajo de \cite{simon2020barycenters}.

Con respecto a la estimación de la distribución real y a la aproximación de variedades, la WGAN realiza estas dos tareas bastante bien, gracias a su componente adversaria. Sin embargo, este no posee un codificador para poder proyectar sobre la variedad de imágenes. Por el otro lado, el WAE posee este codificador, pero no es tan bueno aproximando la distribución real y la variedad de imágenes. Cada una de estas estructuras cumplen bien con una de las tareas, pero no están diseñadas para cumplir con ambas.

Por esta razón, se propone un modelo que unifique a la WGAN y al WAE, de manera que se pueda estimar la distribución real y proyectar sobre la variedad de imágenes. A continuación se detalla la estructura de este modelo.

\section{Deducción de la arquitectura}\label{sec:deduccion-arquitectura-wae-wgan}  % MARK: - Deducción de la arquitectura

Como se menciona en los capítulos anteriores, la WGAN y el WAE tienen tareas similares: buscan minimizar una estimación de la distancia de Wasserstein entre una distribución de referencia $\Prob_X$ y una distribución generada $\Prob_G$. En particular, la WGAN lo realiza utilizando la $1$-distancia de Wasserstein, mientras que la WAE lo puede hacer con cualquier distancia definida a través de la función de costo $c$.

Por este motivo, la deducción del algoritmo es simple: dejar que la WGAN entrene la generadora y la función crítica, de manera que la generadora ya esté bastante cerca de la distribución real. Luego, se puede utilizar la generadora de la WGAN como decodificadora del WAE, y entrenar esta última arquitectura (donde la generadora ya está más cerca de parecerse a la distribución real) para que el codificador sea capaz de proyectar sobre la variedad de imágenes. Es importante que la función de costo para el WAE sea $c(x, y) = |x - y|$ para que se esté calculando la $1$-distancia de Wasserstein, y la distribución generadora converja en la misma topología en la que la WGAN lo hace.

Por este motivo, el algoritmo propuesto es el siguiente:

\begin{algorithm}[H]
  \caption{Entrenamiento de una WAE-WGAN}\label{alg:WAE-WGAN}
  \begin{algorithmic}[1]
    \Require Tamaño del batch $N$, número de iteraciones para el discriminador $N_d$ y los parámetros de penalización $\lambda_{\mathrm{WGAN}}$ y $\lambda_{\mathrm{WAE}}$.
    \State Inicializar los parámetros de la generadora $G_\theta$ y la función crítica $f_\omega$.
    \While{$\theta$ no ha convergido}
    \State // {Entrenamiento de la función crítica}
    \For{$t=1,\ldots,N_d$}
    \State Muestrear $\{x_i\}_{i=1}^{N} \sim \Prob_X$ desde el conjunto de entrenamiento.
    \State Muestrear $\{z_i\}_{i=1}^{N} \sim \Prob_Z$ desde el espacio latente.
    \State $\tilde x \gets (G_\theta(z_i))_{i=1}^{N}$.
    \State $\cL_{\mathrm{critic}} \gets
      \frac{1}{N} \sum_{i=1}^{N} f_{\omega}(\tilde x_i) - \frac{1}{N} \sum_{i=1}^{N} f_{\omega}(x_i) + \lambda_{\mathrm{WGAN}} \cdot \Call{penalty}{\omega, x, \tilde x}$
    \State Actualizar $f_{\omega}$ por medio de descenso de gradiente en $\pdv{\omega} \cL_{\mathrm{critic}}$.
    \EndFor
    \State // {Entrenamiento de la generadora por parte de la WGAN}
    \State Muestrear $\{z_i\}_{i=1}^{N} \sim \Prob_Z$ desde el espacio latente.
    \State $\cL_{\mathrm{gen}} \gets - \frac{1}{N}\sum_{i=1}^{N} f_\omega(G_\theta(z_i))$
    \State Actualizar $G_\theta$ por medio de descenso de gradiente en $\pdv{\theta} \cL_{\mathrm{gen}}$.
    \State // {Entrenamiento de la WAE}
    \State Muestrear $\{z_i\}_{i=1}^{N} \sim \Prob_Z$ desde el espacio latente.
    \State Muestrear $\tilde z_i \sim \ProbQ_\varphi(\dd z \mid x_i)$ para $i=1\dots N$.
    \State Obtener $\tilde x_i \gets G_\theta(\tilde z_i)$ para $i=1\dots N$.
    \State $\cL_{\mathrm{AE}} \gets \frac{1}{N}\sum_{i=1}^{N} |x_i - \tilde x_i| + \lambda_{\mathrm{WAE}} \cdot \Call{similarity}{z, \tilde z}$
    \State Actualizar $\ProbQ_\varphi$ y $G_\theta$ por medio de descenso de gradiente en $\pdv{\varphi} \cL_{\mathrm{AE}}$ y $\pdv{\theta} \cL_{\mathrm{AE}}$.
    \EndWhile
  \end{algorithmic}
\end{algorithm}

Donde la función $\Call{penalty}{\omega, x, \tilde x}$ es alguna penalización para asegurar que $f_\omega\in \Lip_1(\cX)$ y $\Call{similarity}{z, \tilde z}$ es alguna función de similitud entre densidades (como MMD o alguna otra métrica/distancia) entre las distribuciones $\Prob_Z$ y $\ProbQ_{Z, \varphi} \eqdef \Exp_{X\sim \Prob_X}[\ProbQ_\varphi(\dd z \mid X)]$. Definir el algoritmo de esta manera permite tener una mayor generalidad a la hora de utilizarlo, y poder experimentar con distintas funciones de penalización y similitud.

Notar que en el Algoritmo~\ref{alg:WAE-WGAN} realiza alternadamente tres procesos. Primero, un entrenamiento de la función crítica. Luego, un entrenamiento de la generadora utilizando una estimación de la distancia de Wasserstein en su versión de la WGAN. Finalmente, un entrenamiento de la WAE, el que entrena simultáneamente a la codificadora y generadora. Para estos efectos, la generadora se entrena con el gradiente de la distancia de Wasserstein por ambos algoritmos en una sola iteración.

Además, cuando la función crítica y la decodificadora alcanzan un máximo y mínimo aceptable, respectivamente, entonces la evaluación de estas redes en las funciones de pérdida provee una estimación de la distancia de Wasserstein para ambos casos. Esto resulta útil para monitorear el entrenamiento de la red.

\section{Detalles de la implementación}\label{sec:detalles-implementacion}  % MARK: - Detalles de la implementación

\subsection{WGAN con Gradiente Penalizado Generalizado}\label{ssec:wgan-ggp}  % MARK: - WGAN con Gradiente Penalizado Generalizado

Como función de penalización para la función crítica, se desarrolla una forma novedosa de penalizar el gradiente, de manera que tenga mayor estabilidad. La idea es el de obtener un intermedio de una WGAN con penalización Lipschitz (WGAN-LP) \cite{zhou2018lp} y una WGAN con gradiente penalizado (WGAN-GP) \cite{gulrajani2017improved}.

Dado los parámetros de penalización $\lambda_{GP} \geq 0$ y $\lambda_{LP} \geq 0$, se define la penalización como:
\begin{align*}
    & \lambda_{\mathrm{LP}} \Exp_{\hat x \sim \tau} \Big[\big(||\nabla_{\hat x} f_{\omega}(\hat x)||_2 - 1\big)_{+}^2\Big]
  + \lambda_{\mathrm{GP}} \Exp_{\hat x \sim \tau} \Big[\big( - (||\nabla_{\hat x} f_{\omega}(\hat x)||_2 - 1)\big)_{+}^2\Big] \\
  = & \lambda_{\mathrm{LP}} \Exp_{\hat x \sim \tau} \Big[\big(||\nabla_{\hat x} f_{\omega}(\hat x)||_2 - 1\big)_{+}^2\Big]
  + \lambda_{\mathrm{GP}} \Exp_{\hat x \sim \tau} \Big[\big( ||\nabla_{\hat x} f_{\omega}(\hat x)||_2 - 1 \big)_{-}^2\Big],
\end{align*}
donde $(x)_{+} = \max(0, x)$, $(x)_{-} = -\min(0, x)$ y se usa el hecho de que $(-x)_{+} = (x)_{-}$. De esta manera, la expresión de la ecuación anterior se puede reescribir de la siguiente manera:
\begin{equation}
  \lambda_{\mathrm{LP}} \Exp_{\hat x \sim \tau} [\mathrm{LeakyReLU}_{\rho}(||\nabla_{\hat x} f_{\omega}(\hat x)||_2 - 1)^2],
\end{equation}
con $\rho = \sqrt{\frac{\lambda_{GP}}{\lambda_{LP}}}$ y recordando que
\begin{equation}
  \mathrm{LeakyReLU}_{\rho} (x) \eqdef \begin{cases}
    \rho x & \text{si } x < 0,    \\
    x      & \text{si } x \geq 0.
  \end{cases}
\end{equation}

La razón de porqué es una generalización es porque si se toma $\lambda_{\mathrm{GP}} = \lambda_{\mathrm{LP}}$ entonces se recupera la penalización de la WGAN-GP, y si se toma $\lambda_{\mathrm{GP}} = 0$ entonces se recupera la penalización de la WGAN-LP. De esta manera, se puede tener un control más fino de la penalización del gradiente. Por este motivo, a las WGANs que utilizan esta función de penalización, se les llamará \textit{WGAN con gradiente penalizado generalizado} (WGAN-GGP por sus siglas en inglés).

En los experimentos, se utiliza $\lambda_{\mathrm{LP}} = 10$ y $\lambda_{\mathrm{GP}} = 0.1$ para tener una penalización más fuerte cuando la norma del gradiente es mayor a $1$, pero que apenas penalice (y que más bien, incentive) cuando la norma del gradiente es menor a $1$, de manera que le ayude acercarse a la frontera de las funciones $1$-Lipschitz.

\FM[inline]{Incluir el algoritmo para el cálculo de la penalización}

\subsection{Preparación del Conjunto de Datos}\label{ssec:preparacion-dataset}  % MARK: - Preparación del Conjunto de Datos

Para los experimentos, se utiliza el conjunto de datos $Quick, Draw!$, creada por Google \cite{jongejan2016quick}. Este conjunto de datos ha sido construido a través de un juego en línea donde a los jugadores se les solicita dibujar objetos pertenecientes a una clase particular de objetos en menos de 20 segundos. El conjunto de datos contiene 50 millones de dibujos en escala de grises divididos en cientos tipos de clases.

Para la preparación del conjunto de datos, se ha desarrollado una librería para obtener fácilmente los datos utilizando la API de Google \cite{munoz2023quicktorch}.
En los experimentos se utilizan específicamente las categorías ``Faces'' y ``Smiley Faces''\RED{Estas son categorías. ¿Deberían ir en itálica igual?}
dado que poseen imágenes bastante similares. Sin embargo, en ambos conjuntos existen imágenes anómalas, además de que en la categoría ``Smiley Faces'' existe una mayor diversidad de imágenes. Por este motivo, es necesario realizar un tratamiento del conjunto de datos antes de utilizarlo.

Para la preparación del conjunto de datos, se han utilizado tanto técnicas de agrupamiento (también conocido como \textit{clustering}) como de reducción de dimensionalidad no lineal.
Iterativamente se reduce la dimensionalidad a través de la técnica llamada Aproximación y Proyección de Variedades Uniformes (UMAP, por sus siglas en inglés) \cite{mcinnes2018umap}, para después agrupar o limpiar los datos anómalos sobre dicha agrupación. Para agrupar los datos, se utilizan las técnicas de Agrupación Espacial de Aplicaciones con Ruido Basada en la Densidad (DBSCAN, por sus siglas en inglés) \cite{ester1996density}, o su variante jerárquica (HDBSCAN, por sus siglas en inglés) \cite{campello2013density};\FM{Si queda tiempo, tratar de explicar en qué contexto se utiliza cada uno}
mientras que para limpiar los datos anómalos, se usa el Factor Atípico Local (LOF, por sus siglas en inglés) \cite{breunig2000lof}. Después del tratamiento, en el conjunto de datos quedan $230$K imágenes en total. Las técnicas de agrupación y limpieza fueron\RED{Aquí el verbo habría que cambiarlo} utilizados mediante la librería de \texttt{scikit-learn} \cite{sklearn}.\RED{Siento que el párrafo quedo muy largo}

Se redimensionan las imágenes de $28\times28$ a $32\times32$ píxeles y se normalizan para mejorar el entrenamiento de las redes neuronales. Se realizan diversas técnicas de aumento de datos (también conocido como \textit{data augmentation}), tales como: volteo horizontal (o \textit{random horizontal flip}) con una probabilidad del $0.5$; alejamiento (o \textit{zoom out}) con una probabilidad del $0.3$ en una proporción uniforme entre $[1; 1.25]$ y rotación (o \textit{rotation}\FM{Aquí siento que queda muy claro y que no hay confusión de nombres}) entre los grados $[-10, 10]$.

\RED[inline]{¿Se podría dejar el párrafo anterior como una enumeración mejor?}

\subsection{Arquitectura de la Red}\label{ssec:arquitectura-red}  % MARK: - Arquitectura de la Red

Las redes neuronales son implementadas utilizando la librería de \texttt{PyTorch} \cite{paszke2019pytorch}. Se utilizan redes neuronales convolucionales (CNN por sus siglas en inglés) siguiendo la estrategia de ResNet \cite{he2016deep} para la codificadora $\ProbQ_\varphi$ y la generadora $G_\theta$ y la función crítica $f_\omega$.

\FM[inline]{Ojalá incluir alguna imagen que ilustre la arquitectura}
\FM[inline]{Aquí incluir la arquitectura de la red}

\section{Resultados}\label{sec:resultados-wae-wgan}  % MARK: - Resultados

\FM[inline]{Incluir las imágenes de las diferencias entre las categorías de caras y caras felices}
\FM[inline]{Incluir fotos de la variedad y clusterización}
\FM[inline]{Mostrar gráficos de las funciones de pérdida y de la distancia de Wasserstein}
\FM[inline]{Incluir gráficos de las imágenes generadas y reconstruidas}

\section{Conclusiones}\label{sec:conclusiones-wae-wgan}  % MARK: - Conclusiones

\FM[inline]{Aquí incluir las conclusiones}

