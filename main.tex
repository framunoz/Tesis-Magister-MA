% se puede agregar la opción [english] para 
%  memorias o tesis en inglés (borrando el archivo .aux)
% El documento tiene 3 modos:
% - draft: Para revisión. Incluye notas de revisión, las referencias aparecen como se nombran en el tex e incluye el nombre de los labels al margen. No aparecen las imágenes para acelerar la compilación.
% - (default): También para la revisión, pero sin las notas de revisión y sin los nombres de los labels al margen. Aparecen las imágenes.
% - final: Para la versión final. Sin notas de revisión, sin nombres de labels al margen y con las referencias en el formato correcto. Además, el formato del documento es el solicitado por la biblioteca de la FCFM.
\documentclass[final]{umemoria} 

\addbibresource{bibliografia.bib}

\usepackage[]{pkgs/theoremenvFMG}
\usepackage[swapeps,swapphi]{pkgs/mathcmdFMG}
\usepackage{pkgs/mathprobcmdFMG}
\usepackage[colorinlistoftodos]{pkgs/todonotesFMG}
\usepackage[chapter]{algorithm}
\usepackage{algpseudocode}
\usepackage{float}
\usepackage{lipsum}


\makeatletter
\renewcommand{\ALG@name}{Algoritmo}
\renewcommand{\listalgorithmname}{Índice de \ALG@name s}
\makeatother

\depto{Departamento de Ingeniería Matemática}
\author{Francisco Muñoz Guajardo}
\title{Cálculo del Baricentro de Wasserstein Bayesiano en Conjuntos de Imágenes mediante Descenso del Gradiente Estocástico}

% incluir ambos comandos para una doble titulación
%  o quitar el comando que no aplica
\memoria{Ingeniero Civil Matemático}
\tesis{Magíster en Ciencias de la Ingeniería, Mención Matemáticas Aplicadas}
%\tesis{Doctor en ???} % incluir solo este comando para doctorados

% puede haber varios profesores guía separados por coma;
% pero si es una memoria, solo puede haber un profesor guía
\guia{Joaquín Fontbona Torres} 

% puede haber varios profesores co-guía separados por coma;
% pero si es una memoria, el profesor co-guía será el primer
% integrante de la comisión
%\coguia{Nombre Completo Co-Guía} % incluir en caso de co-guía de *tesis*

%\cotutela{Nombre Institución} % incluir en caso de cotutela

\comision{Nombre Completo Uno,Nombre Completo Dos,Nombre Completo Tres}

%\auspicio{Nombre Institución} % incluir en caso de recibir financiamiento

% tiene que ser el año en que se da el examen de título/grado (defensa)
\anho{2024} % incluir solo para reemplazar el año actual


% MARK: Include only
% \includeonly{
%     % lib/03redesNeuronales,
%     lib/04wganWae,
%     lib/05appBarWass,
% }

\begin{document}

% Acronimos
% \input{lib/acronyms.tex}

\frontmatter
\maketitle

\begin{resumen}
    En el presente trabajo de tesis se expone el desarrollo e implementación del algoritmo de Descenso del Gradiente Estocástico sobre el Espacio de Wasserstein (SGDW) para el cálculo de baricentros de Wasserstein Bayesiano (BWB). La relevancia de este trabajo radica en que, hasta la fecha, no existe implementación del SGDW ni el BWB sobre conjuntos de imágenes.

    Para ello, se comienza con una revisión de la teoría de transporte óptimo y la distancia de Wasserstein, seguida de una explicación de los baricentros de Wasserstein de población y la manera en que se pueden calcular utilizando el SGDW. Posteriormente, y dado que se utilizará como medida a priori de la distribución posterior, se presenta la teoría necesaria de las redes generativas.

    En este contexto, se introduce un método para entrenar redes generativas adversarias basadas en la distancia de Wasserstein, donde también se entrena simultáneamente un codificador para el generador. El codificador será necesario en el futuro para obtener un proyector sobre la variedad de imágenes deseada.

    Se explica la implementación del algoritmo SGDW, se presenta la variación proyectada del algoritmo y se concluye con el cálculo del baricentro de Wasserstein Bayesiano. Cada una de estas secciones está acompañada de experimentos que validan cualitativamente la calidad de las simulaciones.

    Finalmente, se menciona que el resultado de este trabajo es una librería en \textit{Python}, utilizada para desarrollar las simulaciones de este trabajo. Como parte del trabajo futuro, se señala que el SGDW proyectado y la red generativa con codificador requieren especial atención.

\end{resumen}

% opcional: incluir para tesis en inglés;
%  en este caso hay que tener el resumen y abstract
%   en ambos idiomas
%\begin{abstract}
%\lipsum[1-4]
%\end{abstract}

\begin{dedicatoria}
    ``La teoría y la práctica deben
    interactuar constantemente.\\
    La teoría sin la práctica está vacía,
    y la práctica sin la teoría, ciega.''\\
    -- Cross, 1981, p. 110
\end{dedicatoria}

\begin{thanks}
    \lipsum[1-2]
\end{thanks}

\tableofcontents
% \listoftables % opcional
\listoffigures % opcional
\listofalgorithms % opcional
\listoftodos

\mainmatter

% MARK: Capítulos
\chapter{Introducción}

El tema principal de lo que tratará este trabajo de tesis es acerca del \emph{Baricentro de Wasserstein Bayesiano}, concepto introducido por Gonzalo Ríos en el 2020 \cite{rios2020contributions}. A pesar de que el nombre pueda parecer intimidante, este se puede desglosar en tres partes: \emph{baricentro}, \emph{Wasserstein} y \emph{Bayesiano}. A continuación se explicará cada uno de estos conceptos.

Del colegio se conoce que, para cualquier triángulo formado por tres puntos $x_1, x_2, x_3 \in \R^d$, su \emph{baricentro} corresponde al promedio de los puntos, es decir, el punto $x_b$ tal que $x_b =\nolinebreak \frac{x_1 + x_2 + x_3}{3}$. Por tanto, el \emph{baricentro} es sólo otra manera de llamar a lo que típicamente se conoce como el \emph{promedio}.

Sin embargo, este baricentro también se puede calcular resolviendo el siguiente problema de optimización:
\begin{equation}\label{eq:baricentro-problema-optimizacion}
    x_b = \argmin_{x \in \R^d} \sum_{i=1}^3 \frac{1}{3} \norm{x - x_i}_2^2.
\end{equation}
Como es bien sabido, la función definida por $\dist(x, y) \eqdef \norm{x - y}^2$ es una métrica en $\R^d$, y por tanto, el baricentro se puede interpretar como aquel punto que minimiza el promedio de las distancias con respecto a los puntos dados. En otras palabras, es aquel punto que, en promedio, intenta mantenerse lo más cerca posible de los demás puntos, siendo este un \emph{representante} de todos ellos.

Ahora bien, el lector se puede preguntar:
\begin{quotation}
    \textit{¿De verdad fue necesario haber cambiado el promedio por un problema de optimización? ¿No hace el problema más difícil?}
\end{quotation}
Si bien es verdad que hace el problema considerablemente más difícil, la verdadera ganancia es \emph{la generalidad del planteamiento}. En efecto, el promedio utiliza nociones vectoriales (sumas y ponderación), mientras que el problema de optimización definido en \eqref{eq:baricentro-problema-optimizacion} utiliza únicamente nociones métricas (distancias), la cual es una propiedad mucho más general. Esto hace que \emph{el baricentro depende únicamente de la métrica que se esté utilizando, siendo posible que este cambie dependiendo de la métrica}.

Ahora que se ha explicado el concepto de \emph{baricentro}, se pasará a explicar la parte de \emph{Wasserstein} del nombre. Consideremos la situación en la que se tiene una montaña de tierra, y se desea mover esta tierra a un pozo, el cuál posee un volumen igual a la de la montaña. La pregunta es: ¿Cuál es la manera de mover la tierra de la montaña al pozo de la manera más eficiente? La respuesta a esta pregunta la da la teoría de \emph{transporte óptimo}, y a partir de esta rama es que se define la \emph{distancia de Wasserstein}.

Sin entrar en muchos detalles, la distancia de \emph{Wasserstein} es una métrica entre dos medidas de probabilidad: una medida $\mu\in \ProbSpace[\cX]$ que representa la montaña, y otra medida $\nu\in \ProbSpace[\cX]$ que representa el pozo. Intuitivamente, la distancia de Wasserstein \emph{determina el esfuerzo promedio mínimo} que se necesita para mover la tierra de la montaña al pozo.

Con esta idea en mente, si es que la montaña estuviera más lejos del pozo, entonces la distancia de Wasserstein sería mayor. La propiedad de que esta distancia sea sensible a traslaciones espaciales (que muchas veces no la cumplen otras métricas y divergencias) le provee a esta distancia muchas propiedades interesantes.

Son estas propiedades que convierten a la distancia de Wasserstein en la candidata ideal para calcular baricentros. En donde, si se considera un conjunto de medidas de probabilidad $\{\mu_1, \ldots, \mu_n\}$, y denotando a la $p$-distancia de Wasserstein\footnote{La constante $p$ parametriza la distancia de forma similar a las $p$-normas, aunque para esta introducción, no es necesario entrar en detalles.} entre dos medidas $\mu$ y $\nu$ por $\Wasserstein[p]{\mu}{\nu}$, entonces el \emph{baricentro de Wasserstein} se define por
\begin{equation}\label{intro-wasserstein-barycenter}
    \bar\mu \eqdef \argmin_{\mu \in \ProbSpace[\cX]} \sum_{i=1}^n \gamma_i \Wasserstein[p]{\mu}{\mu_i}^p.
\end{equation}
donde $\left\{ \gamma_1, \ldots, \gamma_n \right\}$ es una secuencia de pesos no negativos que suman 1.

Con esta definición, el lector se podría preguntar:
\begin{quotation}
    \textit{¿Qué sucedería si en vez de tener un conjunto \textbf{finito} de medidas $\mu_1, \ldots, \mu_n$, tuviéramos un conjunto \textbf{infinito} de medidas?}
\end{quotation}
la respuesta es que, en ese caso, se ha de considerar una medida de probabilidad sobre medidas de probabilidad: una medida $\Gamma \in \ProbSpace[\ProbSpace[\cX]]$, que cumple el rol de los pesos $\gamma_i$ en la ecuación \eqref{intro-wasserstein-barycenter}. De esta manera, la sumatoria se convierte en una integral, y la definición del baricentro de Wasserstein se generaliza a
\begin{equation}
    \bar\mu = \argmin_{\mu \in \ProbSpace[\cX]} \int_{\ProbSpace[\cX]} \Wasserstein[p]{\mu}{\nu}^p \; {\Gamma(\dd \nu)}.
\end{equation}

Ahora que ya se conoce el concepto del \emph{baricentro de Wasserstein}, se procederá a explicar la componente \emph{Bayesiana}. En el enfoque Bayesiano, es usual llamar por \emph{modelos} a las de medidas de probabilidad, donde el conjunto de modelos se le denota por $\Models$.\FM{Podría dividir el parrafo aquí.}  Supongamos entonces que tenemos una muestra $\left\{ x_1, \ldots, x_n \right\}$ de $n$ datos que pertenecen a un espacio $\cX$.
% que son idéntica\FM{¿Está bien el tilde?}  e independientemente distribuidos de algún modelo $\tilde\mu$ desconocido.
A partir de esta muestra, se desea determinar un modelo $\bar{\mu}$ en $\Models\subseteq \ProbSpace[\cX]$ que mejor explique los datos, como si estos datos hubieran sido generados por $\bar{\mu}$.\FM{No me gusta cómo quedó esta frase, pero la mantendré por mientras.}

Para esto, se considera la verosimilitud de un modelo $\mu$ y que lo denotaremos por $\LikelihoodN[\mu]$, y un prior sobre los modelos $\Pi(\dd \mu)$. En virtud de la regla de Bayes, la \emph{medida posterior} se define por
\begin{equation}
    \Pi_n(\dd \mu) \eqdef \frac{\LikelihoodN[\mu]}{\int_{\ProbSpace[\cX]} \LikelihoodN[\nu] \Pi(\dd \nu)} \Pi(\dd \mu).
\end{equation}
de esta manera, la medida posterior le dará mayor pesos a aquellas medidas que sean más verosímiles con respecto a los datos. Así, el \emph{baricentro de Wasserstein Bayesiano} se define utilizando la medida posterior:
\begin{equation}
    \bar\mu \eqdef \argmin_{\mu \in \Models} \int_{\ProbSpace[\cX]} \Wasserstein[p]{\mu}{\nu}^p \; {\Pi_n(\dd \nu)}.
\end{equation}


Cabe mencionar que, gracias a las grandes propiedades que posee la distancia de Wasserstein y el baricentro de Wasserstein, es que ha sido utilizada en una gran variedad de aplicaciones:
para comparar los histogramas de colores \cite{rubner1998metric}, para comparar imágenes \cite{peleg1989unified}, para restaurar imágenes \cite{lellmann2014imaging}, para sintetizar texturas \cite{tartavel2016wasserstein}, en la clasificación de texto \cite{kusner2015word} y en particular,
en la función de pérdida para las redes generativas adversarias, para aliviar los problemas del desvanecimiento del gradiente y del modo de colapso \cite{arjovsky2017wasserstein}. Este tema se tratará en mayor profundidad más adelante en la tesis.\FM{Me causa ruido que se incluya este parrafo de esta manera.}

Sin embargo, desde la concepción del baricentro de Wasserstein Bayesiano, tan sólo se realizaron algunos experimentos con conjuntos de datos ``de juguete'' (utilizando distribuciones normales, por ejemplo) para estudiar sus propiedades matemáticas y estadísticas, pero no se han visto aplicaciones en conjuntos de datos más complejos (como podría ser en imágenes).

Una razón de esto, es porque obtener la distribución posterior $\Pi_n(\dd \mu)$ puede ser infactible de calcular para conjuntos de datos finitos, incluso si estos son muy grandes. Otro motivo es que, en el caso en que se utiliza un conjunto infinito de modelos, es necesario utilizar el Descenso del Gradiente Estocástico sobre el Espacio de Wasserstein \cite{backhoff2022stochastic}, el cuál es un algoritmo que, de la misma manera, es bastante reciente y no existen herramientas computacionales que la implementen de forma general, y mucho menos para conjuntos de datos complejos.

Por este motivo, el objetivo general de este trabajo de tesis es la implementación eficiente de una librería en \emph{Python} para el cálculo del baricentro de Wasserstein Bayesiano sobre conjuntos de imágenes, utilizando el Descenso del Gradiente Estocástico sobre el Espacio de Wasserstein.

Dicho esto, los objetivos específicos son:
\begin{itemize}
    \item \textbf{OE1}: Obtener aproximaciones de las medidas de probabilidad $\Gamma \in \ProbSpace[\ProbSpace[\cX]]$.
    \item \textbf{OE2}: Implementar el Descenso del Gradiente Estocástico sobre el Espacio de Wasserstein para una medida $\Gamma \in \ProbSpace[\ProbSpace[\cX]]$.
    \item \textbf{OE3}: Aproximar la medida posterior $\Pi_n(\dd \mu)$.
    \item \textbf{OE4}: Calcular el baricentro de Wasserstein Bayesiano.
\end{itemize}
\chapter{Transporte Óptimo de Masas}
En este capítulo se abordará el problema de transporte óptimo, la distancia de Wasserstein, y el problema de los baricentros de Wasserstein. Además, se presentarán algunas propiedades de la distancia de Wasserstein, las cuales serán de utilidad en el desarrollo de este trabajo. La notación y definiciones utilizadas en este capítulo se encuentran basadas en \cite{villani2009optimal} y \cite{peyre2019computational}. Sin embargo, antes de empezar a enunciar definiciones y propiedades, se sentarán la notación y definiciones básicas que se utilizarán a lo largo de este trabajo.

\section{Notación}
 {
  \begin{definition}
      Se definen los siguientes espacios:
      \begin{itemize}
          \item $(\cX, d)$ es un espacio Polaco, si $\cX$ es un espacio métrico, completo y separable.
          \item $\ProbSpace[\cX]$ denotará al conjunto de medidas de probabilidad en $\cX$, utilizando la $\sigma$-álgebra de Borel.
          \item $\ProbSpaceAC[\cX]$ denotará al conjunto de medidas de probabilidad absolutamente continuas con respecto a una medida de referencia $\lambda$ (como por ejemplo, la de Lebesgue o la cuenta puntos), utilizando la $\sigma$-álgebra de Borel.
          \item $\mathcal{C}(\cX)$ denotará al conjunto de funciones continuas en $\cX$.
      \end{itemize}
  \end{definition}

  \begin{definition}
      Se definirá el \emph{simplex} de dimension $n$ como el conjunto de vectores de $\R^{n}$ cuyas componentes suman 1, es decir,
      \begin{equation}
          \Simplex \eqdef \left\{
          x \in [0, 1]^{n} \colon \sum_{i=1}^{n} x_{i} = 1
          \right\},
      \end{equation}
      y a los elementos pertenecientes al simplex se les llamará \emph{vectores de probabilidad}.
  \end{definition}


  %   \begin{definition}
  %       Se denotará al \emph{conjunto de medidas de probabilidad} en $\cX$ por medio de $\ProbSpace$, y al \emph{conjunto de medidas de probabilidad absolutamente continuas} con respecto a una medida de referencia $\lambda$ (como por ejemplo, la de Lebesgue o la cuenta puntos) por medio de $\ProbSpaceAC$.
  %   \end{definition}

  \begin{definition}
      Dados $\mu \in \ProbSpace[\cX]$ y $\nu \in \ProbSpace[\cY]$, se denotará por $\Cpl(\mu, \nu)$ al conjunto de medidas de probabilidad en $\cX \times \cY$ cuyas proyecciones marginales sean $\mu$ y $\nu$, es decir,
      \begin{equation}
          \Cpl(\mu, \nu) \eqdef \left\{
          \pi \in \ProbSpace[\cX \times \cY] \colon \pi(A \times \cY) = \mu(A), \pi(\cX \times B) = \nu(B), \forall A \subseteq \cX, B \subseteq \cY
          \right\}.
      \end{equation}
  \end{definition}

  \begin{definition}
      Para una función medible $T:\cX \to \cY$ se define el \emph{operador push-forward} de $T$ como la aplicación $\Tpf:\ProbSpace[\cX] \to \ProbSpace[\cY]$ que satisface la siguiente relación:
      \begin{equation}
          \label{eq:pushForward}
          \int_{\cX} f(x) \dd{\Tpf\mu(x)} = \int_{\cY} f(T(x)) \dmu[x], \quad \forall f \in \mathcal{C}(\cY),
      \end{equation}
      para toda $\mu \in \ProbSpace[\cX]$. Adicionalmente, el operador push-forward se puede definir como aquel operador que satisface la siguiente relación:
      \begin{equation}
          \label{eq:pushForward2}
          \forall A \subseteq \cY \text{ medible}, \quad \Tpf\mu(A) = \mu(T^{-1}(A)).
      \end{equation}
  \end{definition}

  \begin{remark}
      Se puede notar que $\Tpf$ preserva la positividad y la masa total, es decir, si $\mu \in \ProbSpace[\cX]$, entonces $\Tpf\mu \in \ProbSpace[\cY]$.
  \end{remark}

  \begin{remark}
      Para el caso en que la medida $\mu \in \ProbSpace$ sea una medida discreta\footnote{i.e. $\mu = \sum_{i=0}^{n} m_i \delta_{x_i}$ con $m\in\Simplex$, $x_1,\ldots, x_n\in\cX$ y $\delta_x$ la medida de Dirac en $x$}, entonces el operador $\Tpf$ lo que hará será intercambiar la masa de cada punto de $\cX$ a su imagen en $\cY$, es decir,
      \begin{equation}
          \label{eq:pushForwardDiscreto}
          \Tpf\mu = \sum_{i=0}^{n} m_i \delta_{T(x_i)}.
      \end{equation}

  \end{remark}



 }

\section{El problema de transporte}
 {
  En esta sección se presentará el problema de transporte óptimo clásico, o el problema de Monge. Este problema consiste en encontrar una manera óptima de transportar una medida de probabilidad $\mu$ a otra medida de probabilidad $\nu$. Para esto, se considera un costo de transporte $c(x, y)$, el cual representa el costo de transportar una unidad de masa desde $x$ a $y$. El problema de transporte óptimo consiste en encontrar una manera óptima de transportar la masa de $\mu$ a $\nu$ de manera que se minimice el costo total de transporte. Formalmente, el problema de transporte óptimo se puede definir de la siguiente manera:

  Empezaremos definiendo
 }
\chapter{Redes Neuronales y Modelos Generativos}\label{chap:redes-neuronales-y-modelos-generativos}
{
Este capítulo introduce los conceptos esenciales acerca de los modelos generativos. Para ello, se empieza por definir lo qué es una red neuronal.
\section{Redes Neuronales}\label{sec:redes-Neuronales}
{
En el último tiempo se ha visto un auge en el uso de las redes neuronales en diversas áreas de la ciencia y la tecnología. Esto se debe a que las redes neuronales han demostrado ser muy efectivas en la resolución de problemas complejos, como la clasificación de imágenes, el procesamiento de lenguaje natural, y la generación de texto e imágenes, entre otros. En este capítulo, se introducirán los conceptos básicos de las redes neuronales, aunque no se profundizará en los detalles de su funcionamiento. Para ello, se recomienda al lector revisar la literatura especializada en el tema.

\section{Redes Neuronales Feedforward}\label{sec:redes-neuronales-feedforward}
{
    A pesar del éxito que han tenido las redes neuronales, y que se puede pensar que son modelos muy complejos, en realidad, las redes neuronales son modelos muy simples. La idea detrás de las redes neuronales es la de aproximar alguna función objetivo $f^\ast$, con la intención de realizar alguna clasificación o regresión. Sin embargo, la tarea puede ser más amplia, como el objetivo de esta sección: que la red aprenda a generar muestras realistas de algún conjunto de datos.


}



Las redes neuronales feedforward, también llamadas perceptrones multicapa, son los modelos de \textit{deep learning} por excelencia. La tarea de estas redes es la de aproximar alguna función objetivo $f^\ast$, con la intención de realizar alguna clasificación o regresión, aunque la tarea puede ser más amplia, como el objetivo de esta sección: que la red aprenda a generar muestras realistas de algún conjunto de datos.

A estos modelos se les llama redes neuronales porque están inspiradas en la forma en que las neuronas del cerebro se comunican entre sí: en como una neurona recibe una señal, la procesa, y luego envía esta señal a otras neuronas para procesar información aún más compleja. Esta analogía se puede ver en la Figura~\ref{fig:ejemplo-red-neuronal}.

\begin{figure}[htbp]
    \centering
    \missingfigure{Ejemplo de una red neuronal}
    % \includegraphics[width=0.5\textwidth]{}
    \caption{caption}
    \label{fig:ejemplo-red-neuronal}
\end{figure}


% Estos modelos se llaman feedforward porque la información $\vx$ fluye a través de la red. Mientras que se les denomina redes porque suelen estar representadas por la combinación de varias funciones distintas. Usualmente se les ilustra como grafos dirigidos acíclicos, describiendo cómo las funciones se encuentran compuestas entre sí.

Formalmente, las redes neuronales se definen como una composición de funciones no-lineales $f^{(\ell)}_{\vth_\ell}$. Cada función $f^{(\ell)}_{\vth_\ell}$ se encuentra parametrizada por $\vth_\ell = (\vW_\ell, b_\ell)$, donde $\vW_\ell \in \R^{d_{\ell} \times d_{\ell-1}}$ corresponden a los pesos y $b_\ell \in \R^{d_\ell}$ corresponde al bias. Estas redes se encuentran definidas de la siguiente manera:
\begin{align*}
    f^{(\ell)}_{\vth_\ell} \colon \R^{d_{\ell-1}} & \to \R^{d_\ell}                                                                         \\
    \vx_\ell                                      & \mapsto f^{(\ell)}_{\vth_\ell}(\vx_\ell) = \sigma^{(\ell)}(\vW_\ell \vx_\ell + b_\ell),
\end{align*}
donde $\sigma^{(\ell)}$ corresponde a la función de activación de la capa $\ell$, la cuál generalmente es una función no-lineal.

Una \emph{red neuronal} $f_{\vth}$ se define como la composición de funciones no-lineales $f^{(\ell)}_{\vth_{\ell}}$,
es decir,
\begin{equation}
    f_{\vth}(\vx) = f^{(L)}_{\vth_{L}} \circ f^{(L-1)}_{\vth_{L-1}} \circ \cdots \circ f^{(1)}_{\vth_{1}}(\vx),
\end{equation}
donde los parámetros de la red $f_{\vth}$ son $\vth = (\vW_1, b_1, \vW_2, b_2, \ldots, \vW_L, b_L)$. Cada una de las funciones no-lineales $f^{(\ell)}_{\vth_{\ell}}$ se les llama la capa $\ell$-ésima de la red, siendo el número de capas $L$ la profundidad de la red.

A partir de la ecuación anterior, se puede ver que las funciones de activación $\sigma^{(\ell)}$ deben de ser funciones no-lineales. Pues si no lo fueran, entonces la red sería una composición de funciones lineales, lo que resulta en una única función lineal. El problema que tendría con esto es que no habría ninguna ganancia con hacer la red más profunda, pues todo colapsaría a una única capa.\FM{Este parrafo se pude quitar, no es tan importante}

Para cumplir con el objetivo de \emph{aprender}, la red neuronal debe de minimizar una función de pérdida $\cL(\vth; \vx, \vy)$, que mide el error que ha cometido la red neuronal al predecir una salida $\vy$ a partir de una entrada $\vx$ utilizando el parámetro $\vth$. Un ejemplo de función de pérdida puede ser el error cuadrático medio:
\begin{equation}
    \cL(\vth; \vx, \vy)
    = \frac{1}{N}
    \sum_{i=1}^{N}
    \|\vy_i - f_{\vth}(\vx_i)\|^2.
\end{equation}
Por tanto, para que la red neuronal aprenda, se debe de resolver el siguiente problema de optimización:
% de forma que la red \emph{aprenda} se reduce en intentar resolver el siguiente problema de optimización:
\begin{align*}
    \min_{\vth} \mathcal{L}(\vth; \vx, \vy)
    %  & = \min_{\vth} \mathcal{L}(\vth; \vx, \vy)            \\
    %   & = \min_{\vth} \mathcal{L}(\vth; \vx, f_{\vth}(\vx)).
\end{align*}


% . Cada función $f^{(\ell)}_{\vth_\ell}$ se encuentra parametrizada por $\vth_\ell = (\vW_\ell, b_\ell)$, y la red neuronal $f_\vth$ se encuentra parametrizada por $\vth = (\vW_{1}, b_1, \ldots, \vW_L, b_L)$. Escrito en términos matemáticos:

% La red neuronal $f_\vth$ correspondería entonces a una composición de funciones no-lineales $f^{(\ell)}_{\vth_\ell}$, donde cada función $f^{(\ell)}_{\vth_\ell}$ se encuentra parametrizada por $\vth_\ell = (\vW_\ell, b_\ell)$, y la red neuronal $f_\vth$ se encuentra parametrizada por $\vth = (\vW_{1}, b_1, \ldots, \vW_L, b_L)$. Escrito en términos matemáticos:




% La red neuronal $f_\theta$ se encuentra parametrizada por $\theta = (W_{1}, b_1, \ldots, W_L, b_L)$. Escrito en términos matemáticos:
% y la red neuronal $f_\theta$ se encuentra parametrizada por $\theta = (W_{1}, b_1, \ldots, W_L, b_L)$. Escrito en términos matemáticos:

% Los perceptrones multicapa (MLP) corresponden a una parte fundamental al área del Deep Learning. El objetivo de un MLP es el de aproximarse a alguna función $f^\ast$. Por ejemplo, para un clasificador, $y = f^\ast(\vx)$ mapea una entrada $\vx$ a una categoría $y$. Una MLP define
%  Un MLP define una familia de funciones parametrizadas por $\theta$ que denotaremos como $f_\theta$. Por ejemplo, un MLP con una sola capa oculta y una función de activación $\sigma$ es definido como:
% Una red neuronal puede ser entendida como una función $f_\theta$, que es definida por medio de composición de funciones no-lineales $f^{(\ell)}_{\theta_\ell}$. Cada función $f^{(\ell)}_{\theta_\ell}$ se encuentra parametrizada por $\theta_\ell = (W_\ell, b_\ell)$, y la red neuronal $f_\theta$ se encuentra parametrizada por $\theta = (W_{1}, b_1, \ldots, W_L, b_L)$. Escrito en términos matemáticos:
% \begin{equation}
%     f_\theta(x) = f^{(L)}_{\theta_{L}} \circ f^{(L-1)}_{\theta_{L-1}} \circ \cdots \circ f^{(1)}_{\theta_{1}} (x).
% \end{equation}

% Cada función $f^{(\ell)}_{\theta_{\ell}}$ está definida desde un espacio de partida $\R^{d_{\ell-1}}$ a un espacio de llegada $\R^{d_\ell}$. Estas funciones se encuentran definidas por medio de:
% \begin{equation}
%     f^{(\ell)}_{\theta_{\ell}} (x_\ell) = \sigma^{(\ell)} (W_\ell \cdot x_\ell + b_\ell),
% \end{equation}
% donde $x_\ell \in \R^{d_{\ell-1}}$ es la entrada de la función, $\sigma^{(\ell)}$ es una función no-lineal, también llamada \emph{función de activación}, $\theta_\ell = (W_\ell, b_\ell)$ son los parámetros de la función, con $W_\ell \in \R^{d_{\ell} \times d_{\ell-1}}$ los \emph{pesos} y $b_\ell \in \R^{d_\ell}$ el \emph{bias}.

\section{Redes Neuronales Convolucionales}\label{sec:redes-neuronales-convolucionales}
{
    \FM[inline]{Escribir! (será necesario?)}
}  % end of sec. Redes Neuronales Convolucionales

\section{Redes Generativas Adversarias}\label{sec:redes-generativas-adversarias-GAN}
{
% Analogía ladron-policía
Comencemos imaginando la siguiente situación\footnote{Este ejemplo es una adaptación y fue inspirado de \cite[min. 4:32]{santana2017creando}}:
% Supongamos que hay un ladrón que desea engañar a un policía entregándole un billete falso. El ladrón, que es un inexperto, le entrega una servilleta, con una cara dibujada en ella, y que en el otro lado de la servilleta tiene escrito: ``\emph{esto vale un millón de dólares}''. El policía, que ha sido entrenado en la detección de billetes falsos, revisa el billete para comprobar que, efectivamente, es un billete falso. Sin embargo, en vez de enviar a la cárcel al ladrón, lo que hace es decirle al ladrón cuales fueron sus fallos, y de qué manera puede este mejorar en sus falsificaciones. Por su parte, el policía también se entrena más y más en la detección de billetes falsos, pues puede que en algún momento, el ladrón se vuelva tan bueno en la elaboración de billetes falsos, que llegue a engañar al policía con uno de sus billetes.

% Definición de las GANs, como un problema de clasificación reales-falsas
% Las redes generativas adversarias (GANs) son un tipo de arquitectura de redes neuronales que se basan en la analogía del ladrón y el policía. En este caso, el ladrón es una red neuronal generativa $G$, que se encarga de generar muestras que parezcan reales, y el policía es una red neuronal discriminativa $D$, que se encarga de clasificar las muestras como reales o falsas. En este caso, la red generativa $G$ se entrena para engañar a la red discriminativa $D$, y la red discriminativa $D$ se entrena para detectar las muestras generadas por la red generativa $G$.
\begin{quotation}
    \textit{Supongamos que hay un ladrón que desea engañar a un policía entregándole un billete falso. El ladrón, que es un inexperto, le entrega una servilleta, con una cara dibujada en ella, y que en el otro lado de la servilleta tiene escrito: ``\emph{esto vale un millón de dólares}''. El policía, que ha sido entrenado en la detección de billetes falsos, revisa el billete para comprobar que, efectivamente, es un billete falso.}

    \textit{Sin embargo, en vez de enviar a la cárcel al ladrón, lo que hace es decirle al ladrón cuales fueron sus fallos, y de qué manera puede este mejorar en sus falsificaciones. Por su parte, el policía también se entrena más y más en la detección de billetes falsos, pues puede que en algún momento, el ladrón se vuelva tan bueno en la elaboración de billetes falsos, que llegue a engañar al policía con uno de sus billetes.}
\end{quotation}

El marco de las Redes Generativas Adversarias (GAN), introducidos por \cite{goodfellow2014generative}, es en esencia, la analogía del ladrón y el policía.
La GAN define un juego donde el objetivo de la \emph{generadora} (el ladrón) es la de generar muestras que parezcan reales, mientras que el objetivo de la \emph{discriminadora} (el policía) es el de clasificar las muestras como verdaderas o falsas.
En este caso, la generadora se entrena para engañar a la discriminadora, y la discriminadora se entrena para detectar la falsedad de las muestras generadas.

Más formalmente, desde el punto de vista de la teoría de juegos, la GAN se define como un juego de dos jugadores: una generadora $\Prob_G$, donde su conjunto de estrategias es $\ProbSpace[\cX]$ (donde $\cX$ es el conjunto de referencia, e.g. $\cX=[0, 1]^{n_1\times n_2}$ para un conjunto de imágenes), y una discriminadora $\Prob_D(\dd y \mid x)$, donde su conjunto de estrategias es el conjunto de kernels Markovianos.\FM{Creo que aquí podría haber un punto aparte en vez de un punto seguido.} Con esto, la GAN es un juego de suma cero con la siguiente función valor objetivo:
\begin{equation}
    \label{eq:gan-objective}
    V(\Prob_G, \Prob_D)
    = \Exp_{X\sim\Prob_G}{\Exp_{Y \sim \Prob_D(\dd y \mid X)} \left[ \ln Y \right]}
    + \Exp_{\tilde X\sim\Prob_G}{\Exp_{Y \sim \Prob_D(\dd y \mid \tilde X)} \left[ \ln (1 - Y) \right]},
\end{equation}
donde la generadora busca minimizar la función valor, mientras que la discriminadora busca maximizar esta cantidad. El objetivo final de la generadora es el de encontrar una distribución $\Prob_G$ tal que se aproxime a una distribución de referencia $\Prob_X \in \ProbSpace[\cX] $. Por el otro lado, el objetivo de la discriminadora es el de clasificar las muestras verdaderas y falsas, asignándole valores cercano a 1 y 0 respectivamente.

% En la práctica, la forma de implementar la generadora es a través de un \emph{modelo generativo de espacio latente}. Esto es,
% \begin{equation}
%     \Prob_G (\dd x) \eqdef \int_\cZ \Prob_G(\dd x \mid z) \; \Prob_Z \left( \dd z\right),
% \end{equation}
% donde $\cZ$ es el espacio de las variables latentes, y $\Prob_Z \in \ProbSpace[\cZ] $ es la distribución del espacio latente (típicamente se utiliza distribución Gaussiana o Uniforme). Por simplicidad, el modelo generativo $\Prob_G (\dd x \mid z)$ se mapea de forma determinista a través de $\Prob_G (\dd x \mid z) = \delta_{G(z)}(\dd x)$, a través de una función $G\colon \cZ \to \cX$.

El siguiente teorema nos habla acerca de la naturaleza del discriminador óptimo:

\begin{theorem}[El discriminador óptimo calcula la divergencia de JS]
    \label{thm:gan-optimal-discriminator}
    Para cualquier estrategia de generador $\Prob_G$ fijo, existe un único discriminador óptimo $\Prob_D^\ast$ que maximiza la función objetivo \eqref{eq:gan-objective}, el cuál toma una forma determinista por medio de $\Prob_D^\ast(\dd y \mid x) = \delta_{D^\ast(x)}(\dd y)$, donde $D^\ast(x) = \dv{\Prob_X}{(\Prob_G+\Prob_X)}$. En tal caso, la función valor toma la siguiente forma:
    \begin{equation}
        V(\Prob_G, \Prob_D^\ast) = \JS{\Prob_X}{\Prob_G} - 2 \ln 2.
    \end{equation}
\end{theorem}

Este teorema nos dice que el discriminador óptimo es aquel que calcula la divergencia de Jensen-Shannon entre la distribución de referencia $\Prob_X$ y la distribución generada $\Prob_G$, salvo una constante. Además, nos dice que el discriminador toma una forma determinista, de forma que bastaría con buscar una función (como por ejemplo, una red neuronal) $D\colon\cX\to [0, 1]$ tal que la aproxime.

\begin{proof}
    Dado $X \sim \Prob_X$, por la desigualdad de Jensen se tiene que:
    \begin{equation}
        \Exp_{Y \sim \Prob_D \left( \dd y \mid X \right)} \left[ \ln Y \right]
        \leq \ln \left( \Exp_{Y \sim \Prob_D \left( \dd y \mid X \right)} \left[ Y \right] \right),
    \end{equation}
    el cuál es análogo para $\tilde X \sim \Prob_G$. Por el lema de Doob\FM{Agregar cita! Quizás se podría poner el del JSM o el lema directo}, se sabe que existe una función $D \colon \cX \to [0, 1]$ tal que $\Exp_{Y \sim \Prob_D \left( \dd y \mid X \right)} \left[ Y \right]
        = D(X)$,
    % \begin{equation}
    %     \Exp_{Y \sim \Prob_D \left( \dd y \mid X \right)} \left[ Y \right]
    %     = D(X),
    % \end{equation}
    es decir, que el discriminador toma una forma determinista $\Prob_D \left( \dd y \mid x \right) = \delta_{D(x)}(\dd y)$,
    % centrada en $D(X)$,
    y por tanto, se tiene que:
    \begin{equation}
        V(\Prob_G, \Prob_D) \leq \Exp_{X \sim \Prob_X} \Big[ \ln D(X) \Big] + \Exp_{\tilde X \sim \Prob_G} \Big[ \ln \qty\big(1 - D(\tilde X)) \Big],
    \end{equation}
    y dado que la parte derecha de la desigualdad es una cota superior de la función valor, s.p.g. se puede asumir que el discriminador toma una forma determinista en el óptimo. En tal caso, la función valor toma la forma del lado derecho de la desigualdad anterior.
    % \begin{equation}
    %     V(\Prob_G, \Prob_D) = \Exp_{X \sim \Prob_X} \Big[ \ln D(X) \Big] + \Exp_{\tilde X \sim \Prob_G} \Big[ \ln \qty\big(1 - D(\tilde X)) \Big].
    % \end{equation}

    Tomando $\Prob = \Prob_X + \Prob_G$, es claro que $\Prob_X \ll \Prob$ y $\Prob_G \ll \Prob$, y por tanto, existen las derivadas de Radon-Nikodym:
    \begin{align*}
        \rho_X & = \dv{\Prob_X}{\Prob}, & \rho_G & = \dv{\Prob_G}{\Prob},
    \end{align*}
    donde claramente $\rho_X + \rho_G = 1$. En tal caso, la función valor toma la siguiente forma:
    \begin{equation}
        \label{eq:gan-objective-2}
        V(\Prob_G, \Prob_D) = \int_\cX \big[ \rho_X(x) \ln D(x) + \rho_G(x) \ln \qty(1 - D(x)) \big] \; \Prob(\dd x).
    \end{equation}
    Notemos que la función $y \mapsto a \ln y + b \ln(1-y)$ alcanza su máximo en $y^\ast = \frac{a}{a+b}$, y por tanto, la función $D^\ast$ que maximiza la función valor es:
    \begin{equation}
        D^\ast = \frac{\rho_X}{\rho_X + \rho_G} = \dv{\Prob_X}{(\Prob_X + \Prob_G)} .
    \end{equation}
    Por otro lado, si evaluamos $D^\ast$ en la función valor \eqref{eq:gan-objective-2}, se tiene que:
    \begin{align*}
        V(\Prob_G, \Prob_D^\ast)
         & = \Exp_{X \sim \Prob_X} \left[ \ln \dv{\Prob_X}{\Prob}(X) \right] + \Exp_{\tilde X \sim \Prob_G} \left[ \ln \dv{\Prob_G}{\Prob}(\tilde X) \right] \\
         & = \KL{\Prob_X}{\frac{\Prob_X + \Prob_G}{2}} - \ln 2 + \KL{\Prob_G}{\frac{\Prob_X + \Prob_G}{2}} - \ln 2                                           \\
         & = \JS{\Prob_X}{\Prob_G} - 2 \ln 2,
    \end{align*}
    lo que concluye la demostración.
\end{proof}

Dado que para cada generador $\Prob_G$ existe un único discriminador óptimo $\Prob_D^\ast$, entonces tiene sentido definir la siguiente función:
\begin{equation}
    C(\Prob_G) \eqdef V(\Prob_G, \Prob_D^\ast) = \JS{\Prob_X}{\Prob_G} - 2 \ln 2.
\end{equation}
El siguiente teorema nos dice que existe un único generador óptimo $\Prob_G^\ast$ que minimiza la función $C(\Prob_G)$, y que este corresponde a la distribución de referencia $\Prob_X$:

\begin{theorem}
    El mínimo global de la función $C(\Prob_G)$ se alcanza en $\Prob_G^\ast = \Prob_X$, y el valor mínimo es $C(\Prob_X) = - \ln 4$.
\end{theorem}

\begin{proof}
    Por el Teorema anterior, se tiene que:
    \begin{align*}
        \min_{\Prob_G} \max_{\Prob_D} V(\Prob_G, \Prob_D)
         & = \min_{\Prob_G} V(\Prob_G, \Prob_D^\ast)         \\
         & = \min_{\Prob_G} C(\Prob_G)                       \\
         & = \min_{\Prob_G} \JS{\Prob_X}{\Prob_G} - 2 \ln 2.
    \end{align*}
    Como la divergencia de Jensen-Shannon es estrictamente positiva si $\Prob_G \neq \Prob_X$, y nula si $\Prob_G = \Prob_X$, entonces se tiene que el mínimo global de la función $C(\Prob_G)$ se alcanza en $\Prob_G^\ast = \Prob_X$, y el valor mínimo es $C(\Prob_X) = - \ln 4$. Esto concluye la demostración.
\end{proof}

En la práctica, la forma de implementar la generadora es a través de un \emph{modelo generativo de espacio latente}. Esto es,
\begin{equation}
    \Prob_G (\dd x) \eqdef \int_\cZ \Prob_G(\dd x \mid z) \; \Prob_Z \left( \dd z\right),
\end{equation}
donde $\cZ$ es el espacio de las variables latentes, y $\Prob_Z \in \ProbSpace[\cZ] $ es la distribución del espacio latente (típicamente se utiliza la distribución Gaussiana o Uniforme).

Por simplicidad, el modelo generativo $\Prob_G (\dd x \mid z)$ se mapea de forma determinista a través de $\Prob_G (\dd x \mid z) = \delta_{G(z)}(\dd x)$, a través de una función $G\colon \cZ \to \cX$, la cuál la estimaremos a través de una red neuronal $G_\theta$. Por otro lado, el Teorema~\ref{thm:gan-optimal-discriminator} nos dice que, en el óptimo, el discriminador toma una forma determinista, de forma que bastaría con sólo buscar una función $D\colon\cX\to [0, 1]$, la cual la estimaremos a través de una red neuronal $D_\varphi$.
%  tal que la aproxime. Para ello, utilizaremos una red neuronal $D_\phi$.
% el discriminador óptimo toma una forma determinista, de forma que bastaría con sólo buscar una función $D\colon\cX\to [0, 1]$ tal que la aproxime. En la práctica, esta función $D$ se implementa a través de una red neuronal $D_\phi$.

\FM{Agregar una explicación de las funciones de pérdida para este caso, explicar qué cosas buscan minimizar/maximizar y describir el algoritmo de entrenamiento.}
De esta forma, el algoritmo para estimar los parámetros $\theta$ y $\varphi$ de la generadora y la discriminadora, respectivamente, es el siguiente:

\begin{algorithm}[tbhp]
    \caption{GAN}\label{alg:GAN}
    \begin{algorithmic}
        \Require Tamaño del batch $N$ y número de iteraciones para el discriminador $N_d$.
        % Penalization coefficients $\lambda_1, \lambda_2 > 0$, the number of critic iterations $n_{\text{critic}}$, the batch size $m$.
        \State Inicializar los parámetros de la generadora $G_\theta$ y la discriminadora $D_\varphi$.
        % Initialize the parameters of the encoder $Q_\phi$, \\generator/decoder $G_\theta$ and the critic function $f_\omega$.
        \While{$\theta$ no ha convergido}
        \For{$t=0,\ldots,N_d$}
        \State Muestrear $\{x_i\}_{i=1}^{N} \sim \Prob_X$ desde el conjunto de entrenamiento.
        \State Muestrear $\{z_i\}_{i=1}^{N} \sim \Prob_Z$ desde el espacio latente.
        \State $\cL_{\mathrm{disc}} \gets -\frac{1}{N}\sum_{i=1}^{N} \Big[ \ln D_\varphi(x_i) + \ln \qty\big(1 - D_\varphi(G_\theta(z_i))) \Big]$
        \State Actualizar $D_\varphi$ por medio de descenso de gradiente en $\pdv{\varphi} \cL_{\mathrm{disc}}$.
        % \State $\widehat W^{(1)}_{\omega}(\theta) \gets \frac{1}{m}\sum_{i=1}^{m} f_\omega(x_i) - \frac{1}{m}\sum_{i=1}^{m} f_\omega(G_\theta(z_i))$
        % \State $\cL_{\text{critic}} \gets -\widehat W^{(1)}_{\omega}(\theta) + \lambda_1 \cP(f_\omega)$
        % \State Update $f_\omega$ by descending $\pdv{\omega} \cL_{\text{critic}}$.
        \EndFor
        \State Muestrear $\{z_i\}_{i=1}^{N} \sim \Prob_Z$ desde el espacio latente.
        \State $\cL_{\mathrm{gen}} \gets \frac{1}{N}\sum_{i=1}^{N} \ln \qty\big(1 - D_\varphi(G_\theta(z_i)))$
        \State Actualizar $G_\theta$ por medio de descenso de gradiente en $\pdv{\theta} \cL_{\mathrm{gen}}$.
        % \State Update $G_\theta$ by descending $\pdv{\theta} \widehat W^{(1)}_{\omega}(\theta)$.
        % \State Sample $\tilde z_i \sim Q_\phi(dz | x_i)$ for $i=1,\ldots, m$.
        % \State $\widehat W^{(2)}_{\phi}(\theta) \gets \frac{1}{m}\sum_{i=1}^{m} |x_i - G_\theta(\tilde z_i)|$
        % \State $\cL_{\text{WAE}} \gets \widehat W^{(2)}_{\phi}(\theta) + \lambda_2 \cD(\{z_i\}_{i=1}^{m}, \{\tilde z_i\}_{i=1}^{m})$
        % \State Update $Q_\phi$ by descending $\pdv{\phi} \cL_{\text{WAE}}$.
        % \State Update $G_\theta$ by descending $\pdv{\theta} \widehat W^{(2)}_{\phi}(\theta)$.
        \EndWhile
    \end{algorithmic}
\end{algorithm}

\begin{remark}
    Notemos que este algoritmo lo primero que hace intentar encontrar el óptimo de $\max_D V(G, D)$, el cuál proporciona una aproximación para la divergencia de Jensen-Shannon entre $\Prob_X$ y $\Prob_G$. Luego, se actualiza la generadora $G$ para minimizar la divergencia de Jensen-Shannon entre $\Prob_X$ y $\Prob_G$. Con este procedimiento, la generadora $G$ se aproxima a la distribución de referencia $\Prob_X$ utilizando la divergencia de Jensen-Shannon.
\end{remark}






% El objetivo entonces de la generadora es el de maximizar la función objetivo, mientras que el objetivo de la discriminadora es el de minimizar la función objetivo. En otras palabras, la generadora se entrena para engañar a la discriminadora, y la discriminadora se entrena para clasificar la falsedad de las muestras generadas por la generadora.




% Las Redes Generativas Adversarias (GAN por sus siglas en inglés) son un tipo de arquitectura que se componen de dos partes: un generador $G$ y un discriminador $D$. El objetivo de la generadora $G$ es el de generar muestras que parezcan reales, mientras que el objetivo del discriminador $D$ es el de clasificar las muestras como reales o falsas. En este caso, la red generativa $G$ se entrena para engañar a la red discriminativa $D$, y la red discriminativa $D$ se entrena para detectar las muestras generadas por la red generativa $G$. En la analogía anterior, el ladrón corresponde a la red generativa $G$, y el policía corresponde a la red discriminativa $D$.

% Para lograr este objetivo, la generadora $G$ define una ley $\Prob_G$

% Teorema de la GAN: Convergencia en divergencia Jensen-Shannon

% Ejemplos?
}  % end of sec. Redes Generativas Adversarias



}  % end of sec. Redes Neuronales
}  % end of Chapter Modelos Generativos
\chapter{Unificando la WGAN y el WAE}\label{chap:WAE-WGAN}  % MARK: Unificando la WGAN y el WAE

En capítulos anteriores se explica que, tanto las WGANs como las WAEs aproximan una distribución real $\Prob_X$, además permiten aproximar las variedades, como también proyectar sobre ellas.
Por este motivo, en este trabajo se estima una distribución de imágenes $\Prob_X$ para tener acceso a un muestreo rápido de este, además, para que los baricentros sean naturales, se utiliza una estructura de AE para proyectar sobre la variedad de imágenes, siguiendo el trabajo de \cite{simon2020barycenters}.


Con respecto a la estimación de la distribución real y la aproximación de variedades, la WGAN realiza estas dos tareas bastante bien, gracias a su componente adversaria. Sin embargo, este no posee un codificador (que es necesario para poder proyectar sobre la variedad de imágenes). Por el otro lado, el WAE posee este codificador, pero no es tan bueno generando muestras de la distribución real o aproximando la variedad de imágenes. Cada una de estas estructuras cumplen bien con una de las tareas, pero no están diseñadas para cumplir con ambas.

\FM[inline]{Se podría agregar referencias de trabajos anteriores que tienen un AE adversario, que conectan ambos modelos, pero que no se ha visto una versión Wasserstein, que permite converjer en la top. débil}

Por esta razón, se propone un modelo que unifique a la WGAN y al WAE, de manera que se pueda estimar la distribución real y proyectar sobre la variedad de imágenes. A continuación se detalla la estructura de este modelo.

\section{Preparación del Conjunto de Datos}\label{ssec:preparacion-dataset}  % MARK: - Preparación del Conjunto de Datos

Para los experimentos, se utiliza el conjunto de datos $Quick, Draw!$, creada por Google \cite{jongejan2016quick}. Este conjunto de datos ha sido construido a través de un juego en línea, donde a los jugadores se les solicita dibujar objetos pertenecientes a una clase particular de objetos, en menos de 20 segundos. El conjunto de datos contiene 50 millones de dibujos en escala de grises divididos en cientos de tipos de clases.

Para la preparación del conjunto de datos, se ha desarrollado la librería \textit{Quick, Torch!} \cite{munoz2023quicktorch} para obtener fácilmente los datos utilizando la API de Google.
En los experimentos se utilizan específicamente las categorías ``Faces'' y ``Smiley Faces'', dado que poseen imágenes bastante similares. Sin embargo, en ambos conjuntos existen imágenes anómalas, además de que en la categoría ``Smiley Faces'' existe una mayor diversidad de imágenes. Por este motivo, es necesario realizar un tratamiento del conjunto de datos antes de utilizarlo. Un ejemplo de las imágenes de estas categorías se muestra en la Figura~\ref{fig:quick-draw-caras}.

\begin{figure}[htbp]
    \centering
    \includegraphics[width=0.75\textwidth]{img/quick-draw/caras.jpg}
    \caption{Ejemplo de imágenes de las categorías ``Faces'' y ``Smiley Faces''. Elaboración propia.}
    \label{fig:quick-draw-caras}
\end{figure}


Para la preparación del conjunto de datos, se han utilizado tanto técnicas de agrupamiento (también conocido como \textit{clustering}), como de reducción de dimensionalidad no lineal.
Iterativamente se reduce la dimensionalidad a través de la técnica llamada Aproximación y Proyección de Variedades Uniformes (UMAP, por sus siglas en inglés) \cite{mcinnes2018umap}, para después agrupar o limpiar los datos anómalos sobre dicha agrupación. Para agrupar los datos, se utilizan las técnicas de Agrupación Espacial de Aplicaciones con Ruido Basada en la Densidad (DBSCAN, por sus siglas en inglés) \cite{ester1996density}, o su variante jerárquica (HDBSCAN, por sus siglas en inglés) \cite{campello2013density};
mientras que para limpiar los datos anómalos, se usa el Factor Atípico Local (LOF, por sus siglas en inglés) \cite{breunig2000lof}. Después del tratamiento, en el conjunto de datos quedan $230$K imágenes en total. Las técnicas de agrupación y limpieza fueron utilizadas mediante la librería de \textit{scikit-learn} \cite{sklearn}.

A continuación se ilustran algunas imágenes en las que se utilizan las técnicas mencionadas.
En la Figura~\ref{fig:umap} se puede observar la proyección UMAP en dos dimensiones del conjunto de datos, donde claramente se notan dos agrupaciones. En la Figura~\ref*{fig:umap-clusterizado} se presentan las agrupaciones utilizando DBSCAN. Por último, en la Figura~\ref*{fig:umap-clusters} se presentan las imágenes correspondientes a cada etiqueta de la Figura~\ref*{fig:label2}. Se puede apreciar que en estas figuras se observa una tendencia clara en la forma en que se agrupan las imágenes.

\begin{figure}[H]
    \centering
    \begin{subfigure}[b]{0.45\textwidth}
        \centering
        \includegraphics[height=6cm]{img/cleaning-data/umap.png}
        \caption{Proyección UMAP del conjunto de datos.}
        \label{fig:umap}
    \end{subfigure}
    \hfill
    \begin{subfigure}[b]{0.45\textwidth}
        \centering
        \includegraphics[height=6cm]{img/cleaning-data/umap-clusterizado.png}
        \caption{Proyección UMAP agrupada.}
        \label{fig:umap-clusterizado}
    \end{subfigure}
    \caption{Visualización de la proyección UMAP y su agrupación.}
    \label{fig:umap-example}
\end{figure}

\begin{figure}[H]
    \centering
    \begin{subfigure}[b]{0.45\textwidth}
        \centering
        \includegraphics[width=\textwidth]{img/cleaning-data/label1.png}
        \caption{Imágenes de la etiqueta 0.}
        \label{fig:label1}
    \end{subfigure}
    \begin{subfigure}[b]{0.45\textwidth}
        \centering
        \includegraphics[width=\textwidth]{img/cleaning-data/label2.png}
        \caption{Imágenes de la etiqueta 1.}
        \label{fig:label2}
    \end{subfigure}
    \begin{subfigure}[b]{0.45\textwidth}
        \centering
        \includegraphics[width=\textwidth]{img/cleaning-data/outliers.png}
        \caption{Datos anómalos.}
        \label{fig:outliers}
    \end{subfigure}
    \caption{Imágenes correspondiente a los etiquetados en la Figura~\ref{fig:umap-clusterizado}.}
    \label{fig:umap-clusters}
\end{figure}

\FM[inline]{Finalmente, a las imagenes seleccionadas, se les realiza las siguientes transformaciones: ...}

Se redimensionan las imágenes de $28\times28$ a $32\times32$ píxeles y se normalizan para mejorar el entrenamiento de las redes neuronales. Se realizan diversas técnicas de aumento de datos (también conocido como \textit{data augmentation}), tales como: volteo horizontal aleatorio (o \textit{random horizontal flip}) con una probabilidad del $0.5$; alejamiento aleatorio (o \textit{random zoom out}) con una probabilidad del $0.3$ con un aumento que distribuye como una uniforme de rango $[1; 1.25]$ y rotación aleatoria (o \textit{random rotation}) con un ángulo aleatorio entre $[-10, 10]$.

\RED[inline]{¿Se podría dejar el párrafo anterior como una enumeración mejor?}

\section{Deducción de la arquitectura}\label{sec:deduccion-arquitectura-wae-wgan}  % MARK: - Deducción de la arquitectura

Como se menciona en los capítulos anteriores, la WGAN y el WAE tienen tareas similares: buscan minimizar una estimación de la distancia de Wasserstein entre una distribución de referencia $\Prob_X$ y una distribución generadora $\Prob_G$. En particular, la WGAN lo realiza utilizando la $1$-distancia de Wasserstein, mientras que el WAE lo puede hacer con cualquier distancia definida a través de la función de costo $c$.

Por este motivo, la deducción del algoritmo es simple: dejar que la WGAN entrene la generadora y la función crítica, de manera que la generadora ya esté bastante cerca de la distribución real. Luego, se puede utilizar la generadora de la WGAN como decodificadora del WAE, y entrenar esta última arquitectura (donde la generadora ya está más cerca de parecerse a la distribución real) para que el codificador sea capaz de proyectar sobre la variedad de imágenes. Es importante que la función de costo del WAE sea $c(x, y) = |x - y|$ para que se esté calculando la $1$-distancia de Wasserstein, y la distribución de la decodificadora converja en la misma topología que la WGAN.


Por este motivo, el algoritmo propuesto es el siguiente:

\begin{algorithm}[H]
    \caption{Entrenamiento de una WAE-WGAN, elaboración propia}\label{alg:WAE-WGAN}
    \begin{algorithmic}[1]
        \Require Tamaño del batch $N$, número de iteraciones para el discriminador $N_d$ y los parámetros de penalización $\lambda_{\mathrm{WGAN}}$ y $\lambda_{\mathrm{WAE}}$.
        \State Inicializar los parámetros de la generadora $G_\theta$ y la función crítica $f_\omega$.
        \While{$\theta$ no ha convergido}
        \State // {Entrenamiento de la función crítica}
        \For{$t=1,\ldots,N_d$}
        \State Muestrear $\{x_i\}_{i=1}^{N} \sim \Prob_X$ desde el conjunto de entrenamiento.
        \State Muestrear $\{z_i\}_{i=1}^{N} \sim \Prob_Z$ desde el espacio latente.
        \State $\tilde x \gets (G_\theta(z_i))_{i=1}^{N}$.
        \State $\cL_{\mathrm{critic}} \gets
            \frac{1}{N} \sum_{i=1}^{N} f_{\omega}(\tilde x_i) - \frac{1}{N} \sum_{i=1}^{N} f_{\omega}(x_i) + \lambda_{\mathrm{WGAN}} \cdot \Call{penalty}{\omega, x, \tilde x}$
        \State Actualizar $f_{\omega}$ por medio de descenso de gradiente en $\pdv{\omega} \cL_{\mathrm{critic}}$.
        \EndFor
        \State // {Entrenamiento de la generadora por parte de la WGAN}
        \State Muestrear $\{z_i\}_{i=1}^{N} \sim \Prob_Z$ desde el espacio latente.
        \State $\cL_{\mathrm{gen}} \gets - \frac{1}{N}\sum_{i=1}^{N} f_\omega(G_\theta(z_i))$
        \State Actualizar $G_\theta$ por medio de descenso de gradiente en $\pdv{\theta} \cL_{\mathrm{gen}}$.
        \State // {Entrenamiento de la WAE}
        \State Muestrear $\{z_i\}_{i=1}^{N} \sim \Prob_Z$ desde el espacio latente.
        \State Muestrear $\tilde z_i \sim \ProbQ_\varphi(\dd z \mid x_i)$ para $i=1\dots N$.
        \State Obtener $\tilde x_i \gets G_\theta(\tilde z_i)$ para $i=1\dots N$.
        \State $\cL_{\mathrm{AE}} \gets \frac{1}{N}\sum_{i=1}^{N} |x_i - \tilde x_i| + \lambda_{\mathrm{WAE}} \cdot \Call{similarity}{z, \tilde z}$
        \State Actualizar $\ProbQ_\varphi$ y $G_\theta$ por medio de descenso de gradiente en $\pdv{\varphi} \cL_{\mathrm{AE}}$ y $\pdv{\theta} \cL_{\mathrm{AE}}$.
        \EndWhile
    \end{algorithmic}
\end{algorithm}

Donde la función $\Call{penalty}{\omega, x, \tilde x}$ es alguna penalización para asegurar que $f_\omega\in \Lip_1(\cX)$, y $\Call{similarity}{z, \tilde z}$ es alguna función de similitud entre densidades (como MMD o alguna otra divergencia o distancia) entre las distribuciones $\Prob_Z$ y $\ProbQ_{Z, \varphi} \eqdef \int_{\cX} \ProbQ_\varphi(\bullet \mid x) \Prob_X(\dx)$. Definir el algoritmo de esta manera permite tener una mayor generalidad a la hora de utilizarlo, y poder experimentar con distintas funciones de penalización y similitud.

Notar que el Algoritmo~\ref{alg:WAE-WGAN} realiza alternadamente tres procesos. Primero, un entrenamiento de la función crítica. Luego, un entrenamiento de la generadora, utilizando una estimación de la distancia de Wasserstein en su versión de la WGAN. Finalmente, un entrenamiento del WAE, el que entrena simultáneamente a la codificadora y generadora. Para estos efectos, la generadora se entrena con el gradiente de la distancia de Wasserstein por ambos algoritmos en una sola iteración.


Además, cuando la función crítica y la decodificadora alcanzan un máximo y mínimo aceptable, respectivamente, entonces la evaluación de estas redes en las funciones de pérdida provee una estimación de la distancia de Wasserstein para ambos casos. Esto resulta útil para monitorear el entrenamiento de la red.

\subsubsection{Implementación}\label{sssec:wae-wgan-implementacion}  % MARK: - Implementación

El Algoritmo~\ref{alg:WAE-WGAN} se implementa utilizando la librería de \textit{PyTorch} \cite{paszke2019pytorch} a través de las redes neuronales convolucionales (CNN por sus siglas en inglés) siguiendo la estrategia de ResNet \cite{he2016deep}. La red neuronal se entrena en tres variedades de imágenes diferentes, las cuales han sido obtenidas siguiendo los pasos explicados en la Sección~\ref{ssec:preparacion-dataset}. Algunas muestras de estas variedades se presentan en la Figura~\ref{fig:variedades-imagenes}. Se separa el conjunto de datos en un conjunto de entrenamiento y otro de validación, con un $95\%$ y $5\%$ de las imágenes, respectivamente.

\begin{figure}[H]
    \begin{subfigure}[b]{0.32\textwidth}
        \centering
        \includegraphics[width=\textwidth]{img/cluster/label1.png}
        \caption{1era variedad de imágenes.}
        \label{fig:label1-manifold}
    \end{subfigure}
    \hfill
    \begin{subfigure}[b]{0.32\textwidth}
        \centering
        \includegraphics[width=\textwidth]{img/cluster/label2.png}
        \caption{2da variedad de imágenes.}
        \label{fig:label2-manifold}
    \end{subfigure}
    \hfill
    \begin{subfigure}[b]{0.32\textwidth}
        \centering
        \includegraphics[width=\textwidth]{img/cluster/label3.png}
        \caption{3era variedad de imágenes.}
        \label{fig:label3-manifold}
    \end{subfigure}
    \caption{Variedades de imágenes utilizadas en los experimentos.}
    \label{fig:variedades-imagenes}
\end{figure}

A lo largo del entrenamiento se han registrado las estimaciones de la distancia de Wasserstein para la WGAN y el WAE sobre el conjunto de validación para analizar la convergencia. En la Figura~\ref{fig:wass-dist} se presentan dichas estimaciones. Se puede observar que ambas estimaciones disminuyen a lo largo del entrenamiento, lo que indica que las redes están aprendiendo a aproximar la distancia de Wasserstein correctamente, además de que corrobora que el algoritmo propuesto es estable.

\begin{figure}[H]
    \begin{subfigure}[b]{0.8\textwidth}
        \centering
        \includegraphics[width=\textwidth]{img/wgan-wae/wass-dist-wgan.png}
        \caption{Estimación de la distancia de Wasserstein para la WGAN.}
        \label{fig:wass-dist-wgan}
    \end{subfigure}
    % \hfill
    \begin{subfigure}[b]{0.8\textwidth}
        \centering
        \includegraphics[width=\textwidth]{img/wgan-wae/wass-dist-wae.png}
        \caption{Estimación de la distancia de Wasserstein para el WAE.}
        \label{fig:wass-dist-wae}
    \end{subfigure}
    \caption{Estimación de la distancia de Wasserstein para la WGAN y el WAE.}
    \label{fig:wass-dist}
\end{figure}

Para validar que las redes generativas están aprendiendo a reconstruir y generar imágenes de calidad, se presentan en la Figura~\ref{fig:generacion-imagenes} imágenes reales, decodificadas y generadas por la WGAN y el WAE en las tres variedades de imágenes. Se puede observar que las imágenes generadas por la WAE-WGAN son bastante similares a las imágenes reales, lo que indica que las redes están aprendiendo a aproximar la distribución real. Además, las imágenes decodificadas por el WAE son bastante similares a las imágenes originales, lo que indica que la red está aprendiendo a proyectar sobre la variedad de imágenes simultáneamente.

\begin{figure}[H]
    % Primera variedad
    \begin{subfigure}[b]{0.32\textwidth}
        \centering
        \includegraphics[width=\textwidth]{img/wgan-wae/real1.png}
        \caption{Img. reales.}
        \label{fig:real1}
    \end{subfigure}
    \hfill
    \begin{subfigure}[b]{0.32\textwidth}
        \centering
        \includegraphics[width=\textwidth]{img/wgan-wae/decoded1.png}
        \caption{Img. decodificadas.}
        \label{fig:decoded1}
    \end{subfigure}
    \hfill
    \begin{subfigure}[b]{0.32\textwidth}
        \centering
        \includegraphics[width=\textwidth]{img/wgan-wae/gen1.png}
        \caption{Img. generadas.}
        \label{fig:gen1}
    \end{subfigure}
    % Segunda variedad
    \begin{subfigure}[b]{0.32\textwidth}
        \centering
        \includegraphics[width=\textwidth]{img/wgan-wae/real2.png}
        \caption{Img. reales.}
        \label{fig:real2}
    \end{subfigure}
    \hfill
    \begin{subfigure}[b]{0.32\textwidth}
        \centering
        \includegraphics[width=\textwidth]{img/wgan-wae/decoded2.png}
        \caption{Img. decodificadas.}
        \label{fig:decoded2}
    \end{subfigure}
    \hfill
    \begin{subfigure}[b]{0.32\textwidth}
        \centering
        \includegraphics[width=\textwidth]{img/wgan-wae/gen2.png}
        \caption{Img. generadas.}
        \label{fig:gen2}
    \end{subfigure}
    % Tercera variedad
    \begin{subfigure}[b]{0.32\textwidth}
        \centering
        \includegraphics[width=\textwidth]{img/wgan-wae/real3.png}
        \caption{Img. reales.}
        \label{fig:real3}
    \end{subfigure}
    \hfill
    \begin{subfigure}[b]{0.32\textwidth}
        \centering
        \includegraphics[width=\textwidth]{img/wgan-wae/decoded3.png}
        \caption{Img. decodificadas.}
        \label{fig:decoded3}
    \end{subfigure}
    \hfill
    \begin{subfigure}[b]{0.32\textwidth}
        \centering
        \includegraphics[width=\textwidth]{img/wgan-wae/gen3.png}
        \caption{Img. generadas.}
        \label{fig:gen3}
    \end{subfigure}
    \caption{Imágenes reales, decodificadas y generadas por la WGAN y el WAE en las tres variedades de imágenes. Cada fila corresponde a una variedad de imágenes.}
    \label{fig:generacion-imagenes}
\end{figure}



% \section{Detalles de la implementación}\label{sec:detalles-implementacion}  % MARK: - Detalles de la implementación


% \subsection{WGAN con Gradiente Penalizado Generalizado}\label{ssec:wgan-ggp}  % MARK: - WGAN con Gradiente Penalizado Generalizado

% Como función de penalización para la función crítica, se desarrolla una forma novedosa de penalizar el gradiente, de manera que tenga mayor estabilidad. La idea es obtener un intermedio de una WGAN con penalización Lipschitz (WGAN-LP) \cite{zhou2018lp} y una WGAN con gradiente penalizado (WGAN-GP) \cite{gulrajani2017improved}.

% Dado los parámetros de penalización $\lambda_{GP} \geq 0$ y $\lambda_{LP} \geq 0$, se define la penalización como:
% \begin{align*}
%      & \lambda_{\mathrm{LP}} \Exp_{\hat x \sim \tau} \Big[\big(||\nabla_{\hat x} f_{\omega}(\hat x)||_2 - 1\big)_{+}^2\Big]
%     + \lambda_{\mathrm{GP}} \Exp_{\hat x \sim \tau} \Big[\big( - (||\nabla_{\hat x} f_{\omega}(\hat x)||_2 - 1)\big)_{+}^2\Big] \\
%      & =  \lambda_{\mathrm{LP}} \Exp_{\hat x \sim \tau} \Big[\big(||\nabla_{\hat x} f_{\omega}(\hat x)||_2 - 1\big)_{+}^2\Big]
%     + \lambda_{\mathrm{GP}} \Exp_{\hat x \sim \tau} \Big[\big( ||\nabla_{\hat x} f_{\omega}(\hat x)||_2 - 1 \big)_{-}^2\Big],
% \end{align*}
% donde $(x)_{+} \eqdef \max(0, x)$, $(x)_{-} \eqdef -\min(0, x)$ y se usa el hecho de que $(-x)_{+} = (x)_{-}$. De esta manera, la expresión de la ecuación anterior se puede reescribir de la siguiente forma:
% \begin{equation}
%     \lambda_{\mathrm{LP}} \Exp_{\hat x \sim \tau} [\mathrm{LeakyReLU}_{\rho}(||\nabla_{\hat x} f_{\omega}(\hat x)||_2 - 1)^2],
% \end{equation}
% con $\rho = \sqrt{\frac{\lambda_{GP}}{\lambda_{LP}}}$ y recordando que
% \begin{equation}
%     \mathrm{LeakyReLU}_{\rho} (x) \eqdef \begin{cases}
%         \rho x & \text{si } x < 0,    \\
%         x      & \text{si } x \geq 0.
%     \end{cases}
% \end{equation}

% La razón de por qué es una generalización, es porque si se toma $\lambda_{\mathrm{GP}} = \lambda_{\mathrm{LP}}$ entonces se recupera la penalización de la WGAN-GP, y si se toma $\lambda_{\mathrm{GP}} = 0$ entonces se recupera la penalización de la WGAN-LP. De esta manera, se puede tener un control más fino de la penalización del gradiente. Por este motivo, a las WGANs que utilizan esta función de penalización, se les llamará \textit{WGAN con gradiente penalizado generalizado} (WGAN-GGP por sus siglas en inglés).

% En los experimentos, se utiliza $\lambda_{\mathrm{LP}} = 10$ y $\lambda_{\mathrm{GP}} = 0.1$ para tener una penalización más fuerte cuando la norma del gradiente es mayor a $1$, pero muy baja cuando sea menor a 1. La diferencia entre la penalización mayor y menor es de 100:1, por lo que esta última, funciona como un incentivo, ayudando a la función $f_\omega$ a acercarse a la frontera de las funciones $1$-Lipschitz.

% \FM[inline]{Incluir el algoritmo para el cálculo de la penalización}

% \subsection{Arquitectura de la Red}\label{ssec:arquitectura-red}  % MARK: - Arquitectura de la Red

% \FM[inline]{Cada vez que utilice pytorch lo nombro y lo cito (para que quede más claro)}

% Las redes neuronales son implementadas utilizando la librería de \textit{PyTorch} \cite{paszke2019pytorch}. Se utilizan redes neuronales convolucionales (CNN por sus siglas en inglés) siguiendo la estrategia de ResNet \cite{he2016deep} para la codificadora $\ProbQ_\varphi$, la generadora $G_\theta$ y la función crítica $f_\omega$.


% \FM[inline]{Ojalá incluir alguna imagen que ilustre la arquitectura}
% \FM[inline]{Aquí incluir la arquitectura de la red}

% \section{Resultados}\label{sec:resultados-wae-wgan}  % MARK: - Resultados

% \FM[inline]{Incluir las imágenes de las diferencias entre las categorías de caras y caras felices}
% \FM[inline]{Incluir fotos de la variedad y clusterización}
% \FM[inline]{Mostrar gráficos de las funciones de pérdida y de la distancia de Wasserstein}
% \FM[inline]{Incluir gráficos de las imágenes generadas y reconstruidas}

\section{Conclusiones}\label{sec:conclusiones-wae-wgan}  % MARK: - Conclusiones

A vista de los resultados, se puede concluir que la WAE-WGAN es capaz de generar imágenes de calidad y a la vez entrenar un codificador, de manera que se pueda proyectar sobre la variedad de imágenes. Además, se destaca que la WAE-WGAN es capaz de aproximar la distancia de Wasserstein, lo que indica que la red está aprendiendo a aproximar la distribución real. Por último, se destaca que la WAE-WGAN es capaz de aprender a proyectar sobre la variedad de imágenes, lo que indica que la red está aprendiendo a proyectar sobre la variedad de imágenes.


\chapter{Aplicaciones a los Baricentros de Wasserstein}

\section{Descenso del Gradiente Estocástico sobre el Espacio de Wasserstein}\label{sec:sgdw}  % MARK: SGDW

\FM[inline]{Podría ser de título ``Implementación del SGDW''? ya lo definí anteriormente}

En esta sección se presenta la implementación del SGDW.

\subsection{Interpretación de una Imagen como Medida}\label{ssec:interpr-imagen-medida}  % MARK: - Interpretación de una Imagen como Medida

Recordar que si $\mu\in \ProbSpace[\cX] $ es una medida discreta, entonces esta queda definida de la siguiente forma:
\begin{equation}\label{eq:medida-discreta}
    \mu = \sum_{i=1}^{n} m_i \delta_{x_i},
\end{equation}
donde $m \in \Simplex[n]$ es un vector de probabilidad y $ \left\{ x_1, \dots, x_n \right\} \subseteq \cX $ son sus posiciones. En el caso de una imagen, se puede interpretar como una medida discreta, donde $x_i$ es el $i$-ésimo píxel y $m_i$ es la intensidad de la imagen en el $i$-ésimo píxel. Por lo tanto, se puede muestrear un píxel de una imagen de la misma forma que se muestrea una medida discreta.

\FM[inline]{Es posible que este párrafo este de más.}

Cabe destacar que por definición, realizar un muestreo $x \sim \mu$ es equivalente a muestrear un índice $i \sim \Categorical(m)$ y retornar $x = x_i$. Por este motivo, se implementan las medidas discretas siendo extendidas de la clase \texttt{Categorical} del módulo \texttt{torch.distributions} de \textit{PyTorch}. Del mismo modo, como una imagen es una medida discreta, se extiende la clase para una medida discreta, tal que sea eficiente en memoria y en tiempo de construcción.

\subsection{Implementación del Algoritmo}\label{ssec:implementacion-algoritmo}  % MARK: - Implementación del Algoritmo

Dado que el Algoritmo~\ref{alg:sgdw-clasico} está diseñado para medidas absolutamente continuas, se reinterpreta este algoritmo para que se pueda trabajar con medidas discretas. Para ello, se destaca que la Definición~\ref{def:sgdw} se puede reinterpretar como la $\eta_k$-interpolación geodésica entre las medidas $\mu_k$ y $\tilde \mu_k$, mientras que la Definición~\ref{def:bsgdw} se puede reinterpretar como el baricentro de las medidas $\qty( \mu_k, \tilde\mu_k^{(1)}, \dots, \tilde\mu_k^{(S_k)} )$ con pesos $\qty(1-\eta_k, \frac{\eta_k}{S_k}, \dots, \frac{\eta_k}{S_k}) \in \Simplex[S_k+1]$. De este modo, el Algoritmo~\ref{alg:sgdw-clasico} se extiende de la siguiente manera:
\begin{algorithm}[H]
    \caption{SGDW General}
    \label{alg:sgdw-general}
    \begin{algorithmic}[1]
        \Require Acceso a las muestras de $\Gamma(\dd \mu) \in \ProbSpace[\ProbSpace]$, un esquema de paso $(\eta_k)_k \in [0, 1]^\N$ y un esquema de paso $(S_k)_k \in \N^\N$.
        \State{$k\gets0$}
        \State{Muestrear $\mu_0 \sim \Gamma$}
        \Repeat
        \State{Muestrear $\tilde \mu_k^{(1)}, \dots, \tilde \mu_k^{(S_k)} \simiid \Gamma$}
        \State{$\gamma\gets\qty(1-\eta_k, \frac{\eta_k}{S_k}, \dots, \frac{\eta_k}{S_k})$}
        \State Definir $\mu_k$ como el baricentro de $\qty( \mu_k, \tilde\mu_k^{(1)}, \dots, \tilde\mu_k^{(S_k)} )$ con pesos $\gamma$.
        \State{$k\gets k+1$}
        \Until{un criterio de detención ha sido alcanzado.}
        \State\Return $\mu_k$
    \end{algorithmic}
\end{algorithm}

Como se está trabajando con imágenes, se puede aprovechar su estructurar para calcular una estimación de los baricentros de manera más eficiente, utilizando el algoritmos de Baricentros de Wasserstein Convolucionales \cite{solomon2015convolutional} o su versión Insesgada (\textit{Debiased} en inglés) \cite{janati2020debiased}. Sin embargo, el Algoritmo~\ref{alg:sgdw-general} es lo suficientemente general para ser aplicado a cualquier medida discreta, utilizando la estimación del algoritmo de Sinkhorn \cite{cuturi2013sinkhorn}, por ejemplo. Todos estos métodos de cálculo de baricentros se implementan de manera eficiente utilizando la librería de \textit{Python Optimal Transport} (POT) \cite{flamary2021pot}, donde además esta librería admite la paralelización de los cálculos por medio del GPGPU \cite{owens2008gpu}.

\FM[inline]{Mejorar esto utilizando la medida $\Prob_X$}

Para muestrear a partir de una medida $\Gamma$ a partir de un conjunto de datos $\left\{ \mu_i \right\}_{i=1}^{N} \subseteq \ProbSpace[\cX] $, se puede calcular la medida empírica $\hat \Gamma$ de la siguiente forma:
\begin{equation}\label{eq:medida-empirica}
    \hat \Gamma (\dd \mu) = \frac{1}{N} \sum_{i=1}^{N} \delta_{\mu_i} (\dd \mu).
\end{equation}
Esto corresponde muestrear una imagen cualquiera de forma equiprobable. Otra manera de obtener una $\Gamma$, es a través de un modelo generativo, de manera que muestrear una imagen correspondería a simplemente muestrear un ruido aleatorio $z \sim \Prob_Z$ y aplicar la función generadora $G_\theta(z)$.

\subsection{Resultados y Discusión}\label{ssec:sgdw-resultados-discusion}  % MARK: -- Resultados y Discusión

\FM[inline]{En esta sección se podría incluir el cálculo de un baricentro, tanto del dataset como de la GAN}

\subsection{Conclusiones}\label{ssec:sgdw-conclusiones}  % MARK: -- Conclusiones

\FM[inline]{Insertar aquí alguna conclusión}

\FM[inline]{Una conclusión que se me ocurre es como los dos baricentros, el del conjunto de datos y el de la GAN se parecen, algo que debería de pasar puesto que ambos están aproximando a alguna medida de referencia $\Prob_X$}


% \newpage
\section{SGDW Proyectado}\label{sec:sgdwp}  % MARK: - Section SGDWP


A pesar de que la implementación del SGDW entrega los resultados esperados, el baricentro no parece ser lo suficientemente natural. Por este motivo, se propone una adaptación del SGDW, donde se proyecta el baricentro sobre alguna variedad $\Manifold$ de medidas para que se vea más natural y atractivo.

Para ello, se recapitula la definición de CWB de la Sección~\ref{sec:app-bar-wass-Proyectados}, donde en este trabajo \cite{simon2020barycenters} se explica la forma de proyectar una medida $\mu$ sobre una variedad específica de medidas $\Manifold$, aprendidas por una red generativa.

En este sentido, el conjunto de modelos $\Manifold\subseteq \ProbSpace[\cX] $ correspondería a la variedad generada por la red $G_\theta$, es decir,
\begin{equation}
    \Manifold \eqdef \left\{ G_\theta(z) \in \ProbSpace[\cX] \colon z \in \cZ \right\} \subseteq \ProbSpace[\cX].
\end{equation}
Este cambio puede hacer que el Supuesto~\ref{assump:caso-particular-geodesicamente-convexo} no se cumpla, dado que es posible que $\Manifold$ no sea geodésicamente convexo. Esto provocaría que el Teorema~\ref{thm:convergencia-sgdw} y la Proposición~\ref{prop:convergencia-bsgdw} no garanticen la existencia o la unicidad del baricentro proyectado.

awa


\FM[inline]{Para comprobar estas hipótesis, se realizan los siguientes experimentos.....}
\FM[inline]{Aquí quizás incluir que, por este motivo, se harán experimentos para ver si converge a una única medida o no.}


\subsection{Integración de la Proyección al SGDW}\label{ssec:sgdwp-deduccion-algoritmo}  % MARK: - Integración de la Proyección al SGDW

A pesar de que los autores de \cite{simon2020barycenters} explican que, para obtener los resultados de su artículo utilizan el Algoritmo~\ref{alg:ADMM-CWB}, la realidad es que al revisar su código fuente \cite{imagebar2020simon} se observa que sólo utilizan una iteración del algoritmo anterior (es decir, no utilizan el Lagrangiano Aumentado). De este modo, el algoritmo se simplifica de la siguiente manera: para dos medidas $\mu_0, \mu_1 \in \ProbSpace[\cX] $ y un número $t \in [0, 1]$, empiezan por calcular la $t$-interpolación geodésica de estas medidas, y luego proyectan este baricentro sobre la variedad $\Manifold$ a través del AE.

Motivados por esta simplificación, se propone una adaptación del Algoritmo~\ref{alg:sgdw-general} para que proyecte el baricentro en la variedad $\Manifold$, donde además se agrega el parámetro $n_P\in\N$ para proyectar cada $n_P$ iteraciones del algoritmo. De este modo, se deduce el Algoritmo~\ref{alg:sgdwp} para el SGDW Proyectado.
\begin{algorithm}[H]
    \caption{SGDW Proyectado (SGDWP)}
    \label{alg:sgdwp}
    \begin{algorithmic}[1]
        \Require Acceso a las muestras de $\Gamma(\dd \mu) \in \ProbSpace[\ProbSpace]$, un esquema de paso $(\eta_k)_k \in [0, 1]^\N$, un esquema de paso $(S_k)_k \in \N^\N$, un proyector $P:\ProbSpace[\cX] \to \Manifold\subseteq \ProbSpace[\cX] $ y un número $n_P \in \N$.
        \State{$k\gets0$}
        \State{Muestrear $\mu_0 \sim \Gamma$}
        \Repeat
        \State{Muestrear $\tilde \mu_k^{(1)}, \dots, \tilde \mu_k^{(S_k)} \simiid \Gamma$}
        \State{$\gamma\gets\qty(1-\eta_k, \frac{\eta_k}{S_k}, \dots, \frac{\eta_k}{S_k})$}
        \State Definir $\mu_k$ como el baricentro de $\qty( \mu_k, \tilde\mu_k^{(1)}, \dots, \tilde\mu_k^{(S_k)} )$ con pesos $\gamma$.
        \If{$k \mod n_P = 0$}
        \State{$\mu_k\gets P(\mu_k)$}\Comment{Proyectar $\mu_k$ sobre $\Manifold$}
        \EndIf
        \State{$k\gets k+1$}
        \Until{un criterio de detención ha sido alcanzado.}
        \State{$\mu_k\gets P(\mu_k)$}\Comment{Terminar con una última proyección antes de retornar.}
        \State\Return $\mu_k$
    \end{algorithmic}
\end{algorithm}

\FM[inline]{Este párrafo podría ser parte de las conclusiones? Onda, por el trabajo futuro.}

Cabe destacar que el Algoritmo~\ref{alg:sgdwp} tiene una mecánica similar al Método del Gradiente Proyectado (MPG)~\cite[Secc. 5.1]{optimizacion2022amaya}, puesto que este es un método de optimización con restricciones que, si en una iteración se viola alguna restricción, entonces se proyecta dicha iteración sobre el conjunto de restricciones. Este es el caso del Algoritmo~\ref{alg:sgdwp}, donde se proyecta el baricentro en la variedad $\Manifold$. Sin embargo, se diferencia del MPG, en que el dominio $\Manifold$ puede no ser (geodésicamente) convexo.

\FM[inline]{debería mencionar como se implemnetó el proyector $P$ en más detalle?}


\subsection{Resultados y Discusión}\label{ssec:sgdwp-resultados-discusion}  % MARK: - Resultados y Discusión

\subsection{Conclusiones}\label{ssec:sgdwp-conclusiones}  % MARK: - Conclusiones


% \newpage
\section{Baricentro de Wasserstein Bayesiano}\label{sec:bwb}  % MARK: - Section Baricentro de Wasserstein Bayesiano


Teniendo un conjunto de modelos finito, dígase, $\Models \subseteq \ProbSpace[\cX] $ con $| \Models | = N < + \infty$, un acercamiento ``ingenuo'' para calcular la medida posterior sobre este espacio debría considerar el vector de verosimilitudes $L_n = (\cL_n(\mu))_{\mu \in \Models} \in \R^N$. El problema con este enfoque es que, por definición, es bastante probable que uno de los puntos en que se evalúa $\rho_\mu$ sea $0$. Es decir, si $\exists x_i \in D \colon \rho_\mu(x_i) = 0$, entonces ese modelo tendrá una verosimilitud nula. En la práctica, esto sucede con mucha frecuencia. Por este motivo, se decide tomar otro enfoque.


\subsection{Construcción de la Posterior Usando una GAN}\label{ssec:construccion-posterior}  % MARK: - Construcción de la Posterior Usando una GAN

\RED[inline]{Hay que seguir revisando esta sección}

Como se explicó en secciones anteriores,\FM{Aquí sería bueno incluir las secciones de lo que se habla esto, sguramente el de la GAN y WGAN}
dada una medida de referencia $\Prob_X$\footnote{del cuál se tiene acceso a través de una muestra para obtener una medida empírica $\hat\Prob_X = \frac{1}{N}\sum_{i=1}^{N} \delta_{x_i}$}\FM{Corregir} lo que hacen las redes generativas es aproximarla por medio de un modelo generativo $\Prob_G$. Gracias a esta propiedad, se propone utilizar una GAN como prior para poder calcular la posterior. Cabe destacar que, a pesar de que la idea de utilizar una GAN como prior fue original, ya existía un trabajo que propone algo similar \cite{patel2019bayesian}, sin embargo, en este trabajo de tesis se formalizan estas ideas.

Dada una red generadora $G_\theta \colon \cZ \to \ProbSpace[\cX] $ con una medida en el espacio latente $\Prob_Z$, se propone utilizar como prior a la medida
\begin{equation}
    \Pi^G \eqdef \pf{G_\theta} \Prob_Z.
\end{equation}
De esta manera, la posterior tendría la siguiente forma:
\begin{equation}
    \Pi_n(\dd \mu)
    \eqdef \frac{\cL_n(\mu)}{\int_{\ProbSpace[\cX]} \cL_n(\nu) \; \Pi^G(\dd \nu)} \; \Pi^G(\dd \mu)
    = C^{-1} \cL_n(\mu) \; \Pi^G(\dd \mu),
\end{equation}
donde $C \eqdef \int_{\ProbSpace[\cX]} \cL_n(\nu) \; \Pi^G(\dd \nu)$ es una constante de normalización.

Dada alguna función arbitraria $\Pi_n$-integrable $g$ (como por ejemplo $\mu \mapsto \Wasserstein[p]{\mu}{\nu}^p$), se puede comprobar lo siguiente:
\begin{align}
     & \int_{\ProbSpace[\cX]} g(\mu) \; \Pi_n(\dd \mu)                         \\
     & = C^{-1} \int_{\ProbSpace[\cX]} g(\mu) \cL_n(\mu) \; \Pi^G(\dd \mu)     \\
     & = C^{-1} \int_{\cZ} g(G_\theta(z)) \cL_n(G_\theta(z)) \; \Prob_Z(\dd z) \\
     & = \int_{\cZ} g(G_\theta(z)) \; \Pi^Z_n(\dd z),
\end{align}
donde $\Pi^Z_n(\dd z)$ es la \textit{medida posterior en el espacio latente} y se define por
\begin{equation}
    \Pi^Z_n(\dd z) \eqdef C^{-1} \cL_n(G_\theta(z)) \; \Prob_Z(\dd z).
\end{equation}

Por la Definición~\ref{def:operador-push-forward} del operador push-forward, se concluye la siguiente propiedad de la medida posterior:
\begin{equation}\label{eq:posterior-en-latente}
    \Pi_n = \pf{G_\theta} \Pi^Z_n.
\end{equation}
La forma de interpretar la ecuación~\eqref{eq:posterior-en-latente} es que para obtener un muestreo de la posterior $\Pi_n(\dd \mu)$ basta con hacer un muestreo de la posterior en el espacio latente $\Pi^Z_n(\dd z)$ y después aplicar la red generadora $G_\theta$.

Esto resulta beneficioso, pues delega la tarea de muestrear a partir de la posterior al espacio latente $\cZ$ (el cuál, en la práctica, es $\R^{d_z}$), la cual es mucho más fácil de simular. Por ejemplo, una manera de obtener muestreos a partir de $\Pi^Z_n$, es por medio del método de Markov Chain Monte Carlo (MCMC) \cite{andrieu2003introduction,brooks2011handbook,goodman2010ensemble}.\RED{Revisar este párrafo}






% Esto es, dado una red generadora $G_\theta \colon \cZ \to \ProbSpace[\cX] $, se propone como prior sobre el conjunto de modelos de la siguiente manera:
% \begin{equation}
%     \Pi_G(\dd \mu) \eqdef \pf{G_\theta} \Prob_Z(\dd \mu).
% \end{equation}
% De manera que la posterior tendría la siguiente forma:
% \begin{equation}
%     \Pi_n(\dd \mu)
%     \eqdef \frac{\cL_n(\mu)}{\int_{\ProbSpace[\cX]} \cL_n(\nu) \Pi_G(\dd \nu)} \Pi_G(\dd \mu)
%     = \frac{1}{C} \cL_n(\mu) \Pi_G(\dd \mu),
% \end{equation}
% donde $C \eqdef \int_{\ProbSpace[\cX]} \cL_n(\nu) \Pi_G(\dd \nu)$ es una constante de normalización.


% De este modo, la función valor del problema de optimización de la Definición~\ref{def:baricentroWassersteinBayesiano} se puede reescribir de la siguiente manera:
% \begin{align}
%      & \int_{\ProbSpace[\cX]} \Wasserstein[p]{\mu}{\nu}^p \; \Pi_n(\dd \nu)                              \\
%      & = \frac{1}{C} \int_{\ProbSpace[\cX]} \Wasserstein[p]{\mu}{\nu}^p \cL_n(\nu) \; \Pi_G(\dd \mu)     \\
%      & = \frac{1}{C} \int_{\cZ} \Wasserstein[p]{\mu}{G_\theta(z)}^p \cL_n(G_\theta(z)) \; \Prob_Z(\dd z)
% \end{align}

% Con este enfoque, notamos que la función de probabilidad de densidad de la medida posterior en el espacio latente es:
% \begin{equation}
%     \Pi_n^Z (\dz) \eqdef \frac{1}{C} \cL_n(G_\theta(z)) \; \Prob_Z(\dz).
% \end{equation}
% Y de este modo, la medida posterior en el espacio de los modelos es:

% El objetivo de obtener esta función de distribución, es que nos dice que podemos simular muestras de la posterior $\Pi_n(\dd \mu)$ a través de la red generativa $G_\theta$. En efecto, basta darse cuenta que






\subsection{Resultados y Discusión}\label{ssec:bwb-resultados-discusion}  % MARK: - Resultados y Discusión

\subsection{Conclusiones}\label{ssec:bwb-conclusiones}  % MARK: - Conclusiones


\chapter{Conclusiones Generales y Trabajo Futuro}\label{chap:conclusiones-generales}  % MARK: - Chapter Conclusiones Generales

Para finalizar este trabajo de tesis, se presentan las conclusiones obtenidas a partir de los resultados obtenidos en los capítulos anteriores. En particular, se resumen los objetivos específicos y generales, y se discute el trabajo futuro que se puede realizar a partir de esta investigación.

A nivel general, el resultado de este trabajo de tesis proporciona una librería en \textit{Python}, escrita utilizando \textit{PyTorch}. Esta librería hace uso de la GPU para el cálculo, tanto de baricentros de Wasserstein de población, como de baricentros de Wasserstein Bayesianos utilizando el SGDW. Dicho esto, el objetivo general de este trabajo de tesis se ha cumplido.\RED{Mostrar la meta cumplida -> poner el repo}

Pasando a los objetivos específicos, en la Sección~\ref{ssec:implementacion-algoritmo} se explican dos maneras de estimar la medida $\Gamma \in \ProbSpace[\ProbSpace[\cX]]$ que se requiere para el cálculo del baricentro de Wasserstein Bayesiano. En particular, se proponen dos maneras de estimar la medida $\Gamma$: la primera es utilizar una red generativa, y la segunda es ocupar el conjunto de datos directamente. De esta manera, se cumple el objetivo específico \textbf{OE1}. Por otro lado, en la Sección~\ref{ssec:construccion-posterior} se propone una manera de estimar la medida posterior $\Pi_n(\dd \mu)$ usando una red generativa como distribución a priori, de manera que se cumple el objetivo específico \textbf{OE3}.

Con el fin de cumplir el objetivo específico \textbf{OE2}, en la Sección~\ref*{sec:sgdw} se presenta el Algoritmo~\ref{alg:sgdw-general}, que es una adaptación del Algoritmo~\ref{alg:sgdw-clasico} para el cálculo de baricentros de Wasserstein que no posean densidad con respecto a la medida de Lebesgue. De este modo, se logra implementar el SGDW para una medida $\Gamma$, cumpliendo el objetivo específico \textbf{OE2}.

Por último, cumplidos los objetivos específicos \textbf{OE2} y \textbf{OE3}, se calcula el baricentro de Wasserstein Bayesiano en la Sección~\ref{ssec:calc-bwb} para distintos valores de $n$, cumpliendo el objetivo específico \textbf{OE4}.

Adicionalmente, en el transcurso de la tesis se ha desarrollado una manera novedosa de entrenar redes generativas adversarias basadas en la distancia de Wasserstein. Esto, para que el generador posea un codificador, con el objetivo de obtener un proyector sobre la variedad de imágenes deseada. Sin embargo, como este no ha sido el enfoque principal de la tesis, es necesario realizar un estudio más profundo para determinar si este enfoque es efectivo, además de acompañar los resultados visuales con métricas que permitan comparar este método con otros del estado del arte.

Con respecto al descenso del gradiente estocástico, en la Sección~\ref{sec:sgdwp} se ha propuesto una extensión del algoritmo SGDW a una versión proyectada, de manera que el resultado del baricentro tenga un aspecto más natural. Esta extensión del algoritmo abre otras posibilidades de estudio, pues aún falta calibrar los parámetros efectivos. A pesar de esto, los resultados preliminares resultan prometedores.





% ver https://www.overleaf.com/learn/latex/Glossaries
% \input{lib/glosario.tex} % opcional

\nocite{*}
%%%%%%%% CITADO ANTIGUO %%%%%%%%%
% \bibliographystyle{unsrtnat}
% \bibliographystyle{IEEEtran}
% \bibliography{bibliografia.bib}

%%%%%%%% CITADO NUEVO %%%%%%%%%
\printbibliography

% opcional ...
\begin{appendices}
    \chapter{Demostraciones Adicionales de los Teoremas de la GAN}\label{chap:demostraciones-adicionales-teos-gan}
{

    La siguiente demostración se basa en \cite{wikipediagan}.

    \begin{proof}[Demostración del Teorema~\ref{thm:gan-optimal-discriminator}]
        Dado $X \sim \Prob_X$, por la desigualdad de Jensen se tiene que:
        \begin{equation}\label{eq:gan-jensen-inequality-1}
            \Exp_{Y \sim \Prob_D \left( \dd y \mid X \right)} \left[ \ln Y \right]
            \leq \ln \left( \Exp_{Y \sim \Prob_D \left( \dd y \mid X \right)} \left[ Y \right] \right).
        \end{equation}
        Análogamente, dado $\tilde X \sim \Prob_G$, se tiene que:
        \begin{equation}\label{eq:gan-jensen-inequality-2}
            \Exp_{Y \sim \Prob_D ( \dd y \mid \tilde X )} \left[ \ln \qty\big(1 - Y) \right]
            \leq \ln \left( \Exp_{Y \sim \Prob_D ( \dd y \mid \tilde X )} \left[ 1 - Y \right] \right).
        \end{equation}
        Por el lema de Doob \cite[ver Cor. 9.4.11]{sanmartin2018teoria}, se sabe que existe una función $D \colon \cX \to [0, 1]$ tal que $\Exp_{Y \sim \Prob_D \left( \dd y \mid X \right)} \left[ Y \right]
            = D(X)$,
        es decir, que el discriminador toma una forma determinista $\Prob_D \left( \dd y \mid x \right) = \delta_{D(x)}(\dd y)$. Utilizando esta propiedad en las desigualdades \eqref{eq:gan-jensen-inequality-1} y \eqref{eq:gan-jensen-inequality-2} y sumando, se tiene que:
        \begin{equation}
            V(\Prob_G, \Prob_D) \leq \Exp_{X \sim \Prob_X} \Big[ \ln D(X) \Big] + \Exp_{\tilde X \sim \Prob_G} \Big[ \ln \qty\big(1 - D(\tilde X)) \Big],
        \end{equation}
        y dado que la parte derecha de la desigualdad es una cota superior de la función valor, s.p.g. se puede asumir que el discriminador toma una forma determinista en el óptimo. En tal caso, la función valor toma la forma del lado derecho de la desigualdad anterior.
        % \begin{equation}
        %     V(\Prob_G, \Prob_D) = \Exp_{X \sim \Prob_X} \Big[ \ln D(X) \Big] + \Exp_{\tilde X \sim \Prob_G} \Big[ \ln \qty\big(1 - D(\tilde X)) \Big].
        % \end{equation}

        Tomando $\Prob = \Prob_X + \Prob_G$, es claro que $\Prob_X \ll \Prob$ y $\Prob_G \ll \Prob$, y por tanto, existen las derivadas de Radon-Nikodym:
        \begin{align*}
            \rho_X & = \dv{\Prob_X}{\Prob}, & \rho_G & = \dv{\Prob_G}{\Prob},
        \end{align*}
        donde claramente $\rho_X + \rho_G = 1$. En tal caso, la función valor toma la siguiente forma:
        \begin{equation}
            \label{eq:gan-objective-2}
            V(\Prob_G, \Prob_D) = \int_\cX \big[ \rho_X(x) \ln D(x) + \rho_G(x) \ln \qty(1 - D(x)) \big] \; \Prob(\dd x).
        \end{equation}
        Notemos que la función $y \mapsto a \ln y + b \ln(1-y)$ alcanza su máximo en $y^\ast = \frac{a}{a+b}$, y por tanto, la función $D^\ast$ que maximiza la función valor es:
        \begin{equation}
            D^\ast = \frac{\rho_X}{\rho_X + \rho_G} = \dv{\Prob_X}{(\Prob_X + \Prob_G)} .
        \end{equation}
        Por otro lado, si evaluamos $D^\ast$ en la función valor \eqref{eq:gan-objective-2}, se tiene que:
        \begin{align*}
            V(\Prob_G, \Prob_D^\ast)
             & = \Exp_{X \sim \Prob_X} \left[ \ln \dv{\Prob_X}{\Prob}(X) \right] + \Exp_{\tilde X \sim \Prob_G} \left[ \ln \dv{\Prob_G}{\Prob}(\tilde X) \right] \\
             & = \KL{\Prob_X}{\frac{\Prob_X + \Prob_G}{2}} - \ln 2 + \KL{\Prob_G}{\frac{\Prob_X + \Prob_G}{2}} - \ln 2                                           \\
             & = \JS{\Prob_X}{\Prob_G} - 2 \ln 2,
        \end{align*}
        lo que concluye la demostración.
    \end{proof}

    \begin{proof}[Demostración del Teorema \ref{thm:gan-optimal-generator}]
        Por el Teorema anterior, se tiene que:
        \begin{align*}
            \min_{\Prob_G} \max_{\Prob_D} V(\Prob_G, \Prob_D)
             & = \min_{\Prob_G} V(\Prob_G, \Prob_D^\ast)         \\
             & = \min_{\Prob_G} C(\Prob_G)                       \\
             & = \min_{\Prob_G} \JS{\Prob_X}{\Prob_G} - 2 \ln 2.
        \end{align*}
        Como la divergencia de Jensen-Shannon es estrictamente positiva si $\Prob_G \neq \Prob_X$, y nula si $\Prob_G = \Prob_X$, entonces se tiene que el mínimo global de la función $C(\Prob_G)$ se alcanza en $\Prob_G^\ast = \Prob_X$, y el valor mínimo es $C(\Prob_X) = - \ln 4$. Esto concluye la demostración.
    \end{proof}

}  % end of Chapter Demostraciones Adicionales de los Teoremas de la GAN
    \chapter{Complementos del SGDW}\label{chap:complemento-sgdw}  % MARK: Complemnetos del SGDW

\RED[inline]{Hay que revisar este párrafo}

En este anexo se presentan algunos detalles adicionales de la Sección~\ref{sec:sgdw} del Capítulo~\ref{chap:app-bar-wass}. A decir, algunos parámetros adicionales, e imágenes de las iteraciones del algoritmo.

\section{Parámetros Adicionales}\label{sec:compl-sgdw-params-adicional}  % MARK: - Parámetros adicionales

\RED[inline]{Hay que revisar esta sección}

En el Algoritmo~\ref{alg:sgdw-general} se menciona que este debe de recibir como entrada los parámetros $(\eta_k)_k \in [0, 1]^{\N}$ y $(S_k)_k \in \N_\ast^\N$. Sin embargo, no se menciona la manera de cálcular el baricentro en el paso 6. del algoritmo. En este trabajo, se utiliza una estimación del baricentro. Método el cuál también posee sus propios parámetros.

En los experimentos, se utiliza el método de Baricentros de Wasserstein Convolucionales Insesgados \cite{janati2020debiased}, implementada en la librería de POT \cite{flamary2021pot}. En particular, se utiliza la función \texttt{convolutional\_barycenter2d\_debiased} del módulo \texttt{ot.bregman} de POT con los siguientes parámetros:
\begin{itemize}
    \item \texttt{reg} fijado a $10^{-2}$.
    \item \texttt{numItermax} fijado a $10,000$.
    \item \texttt{stopThr} fijado a $10^{-3}$.
\end{itemize}

Estos parámetros han sido fijados de manera heurística, y no se ha realizado un estudio de sensibilidad de los mismos.

\section{Iteraciones del Algoritmo}\label{sec:compl-sgdw-iters}  % MARK: - Iteraciones del Algoritmo

\RED[inline]{Hay que revisar esta sección}

Adicional a los baricentros presentados en la Figura~\ref{fig:barycenters}, se han registrado las primeras y últimas iteraciones del cálculo de los baricentros. En partícular, se presenta la primera medida del muestreo de $\Gamma$, que corresponde a la línea 4. del  Algoritmo~\ref{alg:sgdw-general}; y seguidamente se presenta el paso del algoritmo, que corrresponde a la línea 6. de dicho algoritmo.

% % MARK: - Empirica
% \begin{figure}[htbp]
%     \centering
%     \begin{subfigure}[b]{0.75\textwidth}
%         \includegraphics[width=\textwidth]{img/sgdw-iters/first-iters-DS.pdf}
%         \caption{Primeras iteraciones.}
%         \label{fig:first-iters-DS}
%     \end{subfigure}
%     \begin{subfigure}[b]{0.75\textwidth}
%         \includegraphics[width=\textwidth]{img/sgdw-iters/last-iters-DS.pdf}
%         \caption{Últimas iteraciones.}
%         \label{fig:last-iters-DS}
%     \end{subfigure}
%     \caption{Iteraciones del cálculo del baricentro de $\hat\Prob_X$.}
%     \label{fig:iters-DS}
% \end{figure}

% % MARK: - GAN
% \begin{figure}[htbp]
%     \centering
%     \begin{subfigure}[b]{0.75\textwidth}
%         \includegraphics[width=\textwidth]{img/sgdw-iters/first-iters-GAN.pdf}
%         \caption{Primeras iteraciones.}
%         \label{fig:first-iters-GAN}
%     \end{subfigure}
%     \begin{subfigure}[b]{0.75\textwidth}
%         \includegraphics[width=\textwidth]{img/sgdw-iters/last-iters-GAN.pdf}
%         \caption{Últimas iteraciones.}
%         \label{fig:last-iters-GAN}
%     \end{subfigure}
%     \caption{Iteraciones del cálculo del baricentro de $\tilde\Prob_X$.}
%     \label{fig:iters-GAN}
% \end{figure}

% % MARK: - Empirica Batched
% \begin{figure}[htbp]
%     \centering
%     \begin{subfigure}[b]{0.75\textwidth}
%         \includegraphics[width=\textwidth]{img/sgdw-iters/batch-first-iters-DS.pdf}
%         \caption{Primeras iteraciones.}
%         \label{fig:batch-first-iters-DS}
%     \end{subfigure}
%     \begin{subfigure}[b]{0.75\textwidth}
%         \includegraphics[width=\textwidth]{img/sgdw-iters/batch-last-iters-DS.pdf}
%         \caption{Últimas iteraciones.}
%         \label{fig:batch-last-iters-DS}
%     \end{subfigure}
%     \caption{Iteraciones del cálculo del baricentro de $\hat\Prob_X$ en su versión por lotes.}
%     \label{fig:batch-iters-DS}
% \end{figure}

% % MARK: - GAN Batched
% \begin{figure}[htbp]
%     \centering
%     \begin{subfigure}[b]{0.75\textwidth}
%         \includegraphics[width=\textwidth]{img/sgdw-iters/batch-first-iters-GAN.pdf}
%         \caption{Primeras iteraciones.}
%         \label{fig:batch-first-iters-GAN}
%     \end{subfigure}
%     \begin{subfigure}[b]{0.75\textwidth}
%         \includegraphics[width=\textwidth]{img/sgdw-iters/batch-last-iters-GAN.pdf}
%         \caption{Últimas iteraciones.}
%         \label{fig:batch-last-iters-GAN}
%     \end{subfigure}
%     \caption{Iteraciones del cálculo del baricentro de $\tilde\Prob_X$ en su versión por lotes.}
%     \label{fig:batch-iters-GAN}
% \end{figure}



    \chapter{Complementos del SGDWP}\label{anx:sgdwp}  % MARK: Complementos del SGDWP

\section{Iteraciones Adicionales}\label{sec:sgdwp-iteraciones-adicionales}  % MARK: - Iteraciones Adicionales

A continuación se presentan las primeras y últimas iteraciones que pasaron los baricentros proyectados de la Sección~\ref{sec:sgdwp}. En particular, los baricentros de las Figuras~\ref{fig:bar-SGDWP-pe}.
Recordar que en este caso se utilizó los parámetros $n_P\in\left\{ 1, 3, 5, 10 \right\}$.

\begin{figure}[htbp]
    \centering
    \begin{subfigure}[b]{0.8\textwidth}
        \centering
        \includegraphics[width=\textwidth]{img/sgdwp-pe-iters/first-iters-pe-01.pdf}
        \caption{Primeras iteraciones del baricentro de la Fig.~\ref{fig:bar-SGDWP-pe-01}}
        \label{fig:first-iters-pe-01}
    \end{subfigure}
    \begin{subfigure}[b]{0.8\textwidth}
        \centering
        \includegraphics[width=\textwidth]{img/sgdwp-pe-iters/last-iters-pe-01.pdf}
        \caption{Últimas iteraciones del baricentro de la Fig.~\ref{fig:bar-SGDWP-pe-01}}
        \label{fig:last-iters-pe-01}
    \end{subfigure}
    \caption{Iteraciones del cálculo del baricentro de la Fig.~\ref{fig:bar-SGDWP-pe-01}.}
    \label{fig:iters-pe-01}
\end{figure}

\begin{figure}[htbp]
    \centering
    \begin{subfigure}[b]{0.8\textwidth}
        \centering
        \includegraphics[width=\textwidth]{img/sgdwp-pe-iters/first-iters-pe-03.pdf}
        \caption{Primeras iteraciones del baricentro de la Fig.~\ref{fig:bar-SGDWP-pe-03}}
        \label{fig:first-iters-pe-03}
    \end{subfigure}
    \begin{subfigure}[b]{0.8\textwidth}
        \centering
        \includegraphics[width=\textwidth]{img/sgdwp-pe-iters/last-iters-pe-03.pdf}
        \caption{Últimas iteraciones del baricentro de la Fig.~\ref{fig:bar-SGDWP-pe-03}}
        \label{fig:last-iters-pe-03}
    \end{subfigure}
    \caption{Iteraciones del cálculo del baricentro de la Fig.~\ref{fig:bar-SGDWP-pe-03}.}
    \label{fig:iters-pe-03}
\end{figure}

\begin{figure}[htbp]
    \centering
    \begin{subfigure}[b]{0.8\textwidth}
        \centering
        \includegraphics[width=\textwidth]{img/sgdwp-pe-iters/first-iters-pe-05.pdf}
        \caption{Primeras iteraciones del baricentro de la Fig.~\ref{fig:bar-SGDWP-pe-05}}
        \label{fig:first-iters-pe-05}
    \end{subfigure}
    \begin{subfigure}[b]{0.8\textwidth}
        \centering
        \includegraphics[width=\textwidth]{img/sgdwp-pe-iters/last-iters-pe-05.pdf}
        \caption{Últimas iteraciones del baricentro de la Fig.~\ref{fig:bar-SGDWP-pe-05}}
        \label{fig:last-iters-pe-05}
    \end{subfigure}
    \caption{Iteraciones del cálculo del baricentro de la Fig.~\ref{fig:bar-SGDWP-pe-05}.}
    \label{fig:iters-pe-05}
\end{figure}

\begin{figure}[htbp]
    \centering
    \begin{subfigure}[b]{0.8\textwidth}
        \centering
        \includegraphics[width=\textwidth]{img/sgdwp-pe-iters/first-iters-pe-10.pdf}
        \caption{Primeras iteraciones del baricentro de la Fig.~\ref{fig:bar-SGDWP-pe-10}}
        \label{fig:first-iters-pe-10}
    \end{subfigure}
    \begin{subfigure}[b]{0.8\textwidth}
        \centering
        \includegraphics[width=\textwidth]{img/sgdwp-pe-iters/last-iters-pe-10.pdf}
        \caption{Últimas iteraciones del baricentro de la Fig.~\ref{fig:bar-SGDWP-pe-10}}
        \label{fig:last-iters-pe-10}
    \end{subfigure}
    \caption{Iteraciones del cálculo del baricentro de la Fig.~\ref{fig:bar-SGDWP-pe-10}.}
    \label{fig:iters-pe-10}
\end{figure}
\end{appendices}
\end{document}